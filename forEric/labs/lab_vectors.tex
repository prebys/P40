\chapter{Vectors}

\section{Introduction}

In this lab, we will use numpy arrays to investigate the properties of
vectors in Euclidean space.

\section{Vector Dot Product and Cross Product }

In this lab, we will use numpy arrays of length 3 to represent vectors
in a Euclidean vector space.  So for example:
\begin{python}
  a = np.array([1.2, -3.2, 2.1])
\end{python}
is how we will represent the vector $\vec{a}$ with $a_x=1.2$,
$a_y=-3.2$, and $a_z=2.1$.  To obtain the components of the vector, we retrieve the appropriate element of the array, so $a_x$ is \pyth{a[0]},
$a_y$ is \pyth{a[1]}, and $a_z$ is \pyth{a[2]}.

We construct the unit vectors $\hat{x}$, $\hat{y}$ and $\hat{z}$ as:
\begin{python}
  xhat = np.array([1, 0, 0])
  yhat = np.array([0, 1, 0])
  zhat = np.array([0, 0, 1])
\end{python}
which provides an alternative way to define vectors:
\begin{displaymath}
\vec{b} = 2\hat{x} - 4\hat{y}+1\hat{z}
\end{displaymath}
using the equivalent python code:
\begin{python}
  b = 2*xhat - 4*yhat + 1*zhat
\end{python}

The dot product of two vectors is defined as:
\begin{displaymath}
\vec{a}\cdot\vec{b} = a_x \, b_x +  a_y \, b_y +  a_z \, b_z.
\end{displaymath}
We can determine the magnitude of vector using the dot product:
\begin{displaymath}
  |\vec{a}| = \sqrt{\vec{a}\cdot\vec{a}}
\end{displaymath}
The cross product of two vectors is defined as:
\begin{displaymath}
  \vec{a}\times\vec{b} =
  (a_y \, b_z -  a_z \, b_y) \; \hat{x}
  + (a_z \, b_x -  a_x \, b_z) \; \hat{y}
  + (a_x \, b_y -  a_y \, b_x) \; \hat{z}
\end{displaymath}

\vskip 0.25cm
\plot Implement a function \pyth{dot(a,b)} which returns the dot product of vectors a and b.  You do not need to use a for loop in your implementation if you do not want to: explicitly calling each of three components is simple and clear.  Confirm that your code is working by calculating $\vec{b}\cdot\hat{x}$, $\vec{b}\cdot\hat{y}$, and $\vec{b}\cdot\hat{z}$. \\


\plot Implement a function \pyth{mag(a)} which returns the magnitude of the vector a by calling your \pyth{dot(a,b)} function.  Show that your function works by comparing the magnitude of $\vec{b}$ return by your function with what you expect from paper and pencil.\\

\plot Implement a function \pyth{cross(a,b)} which returns the cross products of the vectors a and b.  Show that your function works for:
\begin{eqnarray*}
  \hat{x} \times \hat{x} &=& 0\\    
  \hat{x} \times \hat{y} &=& \hat{z}\\
  \hat{x} \times \hat{z} &=& -\hat{y}\\
  \hat{y} \times \hat{x} &=& -\hat{z}\\
  \hat{y} \times \hat{z} &=& \hat{x} \\ 
\end{eqnarray*}


The follow example demonstrates that:
\begin{displaymath}
\vec{a} \cdot \vec{b} \leq |\vec{a}|\,|\vec{b}|
\end{displaymath}
by generating many random vectors and testing:
\begin{python}
TOL = 10*np.finfo(float).eps
fail = 0
for i in range(10000):
    ra = np.random.rand(3)
    rb = np.random.rand(3)
    x = dot(ra,rb)-mag(ra)*mag(rb)
    if (x>TOL):
        fail = fail+1
print("failures:  ", fail)
\end{python}
Notice the use of a tolerance instead of strict comparison with zero
to account for the floating point precision.\\

\begin{plot} Demonstrate\end{plot}
\begin{displaymath}
  |\vec{a} + \vec{b}| < |\vec{a}| + |\vec{b}|
\end{displaymath}

\vskip 0.25cm
\begin{plot} Demonstrate the Jacobi identity:\end{plot}
\begin{displaymath}
 \vec{a} \times (\vec{b} \times \vec{c}) + \vec{b} \times (\vec{c} \times \vec{a}) + \vec{c} \times (\vec{a} \times \vec{b}) = 0
\end{displaymath}

\section{Numpy Tools}

The numpy tools for dot and cross products are \pyth{np.dot} and \pyth{np.cross}
\begin{python}
print(np.dot([1,2,3],[0,0,1]))
print(np.cross([1,0,0],[0,1,0]))
\end{python}

\section{Motion of a Charged Particle in a Magnetic Field}

