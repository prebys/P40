\chapter{Debugging}

In our context, debugging is the process of finding and removing
mistakes, called bugs, from your software.  Singling this process out
is a bit deceptive, it makes it seems distinct from software
development, as if you should write your software, and then debug it.
Indeed many students start this way, but it is a painful and
ineffective approach.  Experienced programs debug {\em while} developing
their code.

The fundamental approach to debugging (which works equally well
outside of programming) is to break every problem down into simple,
well defined parts, and then thoroughly test each part.  When one part
does not work, you break it down into smaller parts.  This process can
be quite simple, such as adding print statements to each step of a
complicated calculation.  It can also be quite advanced, such as when
teams of experienced software developers use automated builds and a
suite of integration tests that validate every proposed change to code
before it is accepted.  Experienced programs still produce bugs, they
just get better at squashing them.

There are a number of well-loved techniques to debugging:
\begin{itemize}
\item Print statements.
\item Start with a simple problem.
\item Test on special cases.
\item Use paper and pencil.
\item Decrease the size.
\item Establish feedback.
\item Write modular code.
\item Maintain unit tests.
\end{itemize}
