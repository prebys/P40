\documentclass[12pt,oneside]{book}

\usepackage[dvips,letterpaper,margin=0.75in,bottom=0.75in]{geometry}
\usepackage{cite}
\usepackage{slashed}
\usepackage{graphicx}
\usepackage{amsmath}
\usepackage{enumitem}
\usepackage{amsthm}
\usepackage{braket}
\usepackage{listings}
\theoremstyle{definition}
\usepackage{pythonhighlight}

\usepackage[american,fulldiode]{circuitikz}
\tikzset{component/.style={draw,thick,circle,fill=white,minimum size =0.75cm,inner sep=0pt}}

\newcounter{plot}[chapter]
\setcounter{plot}{0}

\newcommand{\plot}{
\stepcounter{plot}
\noindent
{\bf $\bigtriangleup$ Jupyter Notebook Exercise \thechapter.\theplot:}
}


\newcounter{pencil}[chapter]
\setcounter{pencil}{0}

\newcommand{\pencil}{
\stepcounter{pencil}
\noindent
{\bf $\bigtriangleup$ Pencil and Paper Exercise \thechapter.\theplot:}
}


%\newtheorem{pencil}{Pencil Problem}[chapter]
%\newtheorem{plot}{Jupyter Notebook}[chapter]


\lstnewenvironment{algorithm}[1][] %defines the algorithm listing environment
{   
    \lstset{ %this is the stype
        mathescape=true,
        frame=tB,
        numbers=left, 
        numberstyle=\tiny,
        basicstyle=\scriptsize, 
        keywordstyle=\color{black}\bfseries\em,
        keywords={,input, output, return, datatype, function, in, if, else, foreach, while, begin, end, repeat, print, calculate,} %add the keywords you want, or load a language as Rubens explains in his comment above.
        numbers=left,
        xleftmargin=.04\textwidth,
        #1 % this is to add specific settings to an usage of this environment (for instnce, the caption and referable label)
    }
}
{}


\begin{document}
\ctikzset{bipoles/thickness=1}
\ctikzset{bipoles/length=.6cm}

\title{Physics 40 Lab Manual}
\author{Michael Mulhearn}
\maketitle

\tableofcontents

\chapter{Installation of Scientific Python}

\section{Introduction}

In this lab, you will install the software which we will be using in
phy40. This is an assignment, and will be graded.  You should submit a
text file containing a log of all the steps you took to install the
software on your computer.  Make this log as specific as possible, an
entry might be:
\begin{verbatim}
Downloaded windows installer from:
 https://repo.anaconda.com/miniconda/Miniconda3-latest-Windows-x86_64.exe
\end{verbatim}  
Keeping this log will also make it easier for you to get help if you
have problems.

If you run into problems, do some research on a web search tool
(Google, for example) to become better informed and to see if you can
overcome the problem on your own before asking for help.  This is an
important technique in getting help with technical problems that will
serve you well even outside of this class.  You will find it more easy
to get useful technical help, from the sort of people most capable of
offering it, when it is clear from your question that you are informed
and have already tried all of the obvious things.  If you are still
stuck after trying to solve the problem for yourself, then contact
your TA or instructor with specific technical details about what is
failing, and include your installation log.

If you do find a problem with these instructions or manage to overcome
a technical problem yourself, make sure to note it in your log and
inform your TA, in case it is helpful for other students.

\section{Installing Miniconda3}

We will be using Miniconda3 based on Python 3.7 for data analysis
using Jupyter notebooks.  Miniconda is a lightweight package which we
can use to install all of the remaining analysis software we will need
in a consistent manner across all different operating systems.

Determine which OS type and version you have on the desktop or laptop
computer that you will be using for your coursework.  The software
here will work under Windows, Linux, or macOS.  It should also work on
all Chromebooks released since 2019, and some earlier Chromebooks.
You should also check whether you have a 32-bit or 64-bit OS (you can
find instructions for how to determine this for your particular OS
version with a Google search.)  Most desktop or laptop computers built
in the last ten years are 64-bit.

If you are using Linux or macOS, then from within a terminal type:
\begin{lstlisting}{language=csh}
 echo $SHELL
\end{lstlisting}
to determine the shell you are using (typically "bash" these days).
Record all of this information in your installation log file.

Once you have determined your OS type and version, follow the
instructions below approprate to your operating system.

\subsection{Installing under Windows}

If you have already installed a version of conda (e.g. Anaconda or
Miniconda) then you do not need to re-install it.  Instead, find the
the Anaconda Prompt in the Application menu and run it.

If you need to install Miniconda3, then download and run the
appropriate installer from:
\begin{verbatim}
https://docs.conda.io/en/latest/miniconda.html#
\end{verbatim}
If prompted, you should choose to:
\begin{itemize}
 \item Accept the license / terms of use.
 \item Install for just the current user, not all users.
\end{itemize}
 Once installed, check that you can run the "Anaconda Prompt". From
the prompt, check that you can run:
\begin{lstlisting}{language=csh}
  conda --version
\end{lstlisting}
and note the output in your installation log.  Then proceed to
Section~\ref{sec:env}.

\subsection{Installing on a Chromebook}

You will need to activate Linux on your Chromebook, according to the instructions here:

\begin{verbatim}
https://www.codecademy.com/articles/programming-locally-on-chromebook
\end{verbatim}

Then follow the insturctions for installing under Linux.  If your
Chromebook predates 2019 and does not support Linux, contact your
instructor for alternative arrangements.

\subsection{Installing Miniconda3 under Linux or macOS}

If you believe you already already have a version of conda installed
(such as miniconda or ananconda) , check by running
\begin{lstlisting}{language=csh}
   conda --version
\end{lstlisting}
If you see something like:
\begin{lstlisting}{language=csh}
  conda 4.9.2
\end{lstlisting}
as output (even if the version is different) then you do indeed already have conda
installed, with the base environment activated, and you can skip ahead to
Section~\ref{sec:env}.  If instead you get a message like:
\begin{lstlisting}{language=csh}
   conda: command not found
\end{lstlisting}
then the easiest solution is to simply proceed with these instructions.

To install Miniconda, download the appropriate installer for your OS
here:
\begin{verbatim}
  https://docs.conda.io/en/latest/miniconda.html\#
\end{verbatim}
For macOS, you can choose between a "package" or "bash" version. I
find it easier to follow the bash version, but the package version
will work too. I recommend you make the following choices if prompted:
\begin{itemize}
\item Accept the license / terms of use.
\item Do not install for all users, but just one the current user.
\item Do allow the installer to issue ``conda init''.
\end{itemize}
During the installation, take note of the install location in your log.

After installation with these settings, conda will automatically
activate the ``base'' conda environment.  If this annoys you, as it
does me, or interferes with other software you are using, you can turn
off this agressive behavior with:
\begin{lstlisting}{language=csh}
   conda config --set auto_activate_base false
\end{lstlisting}

Confirm that you have successfully installed conda by typing
\begin{lstlisting}{language=csh}
   conda --version
\end{lstlisting}
Record the output in your installation log, and proceed to Section~\ref{sec:env}.

\section{Installing the Physics 40 Conda Environment}
\label{sec:env}

Make sure your conda is fully up to date with:
\begin{lstlisting}{language=csh}
  conda update conda
\end{lstlisting}
Then follow the prompts, e.g. selecting "y" as needed to update any out-of-date packages.

We'll be using a conda environment specifically for phy40 to avoid
conflicts with any other projects on your computer, and to ensure that
we all have the same software installed.  To create our environment:
\begin{lstlisting}{language=csh}
  conda create -n phy40 python=3.9 numpy scipy matplotlib ipython jupyter
\end{lstlisting}
  
\section{Starting a Jupyter notebook}

This course will make extensive use of the Jupyter Notebook interface
to Scientific Python, which is well suited to academic work (including
independent research) because it combines code with output in
digestable chunks.  Even when the end product is a polished peice of
software, much of the initial code development can be done in the interactive
session that Jupyter Notebooks provide.  

\begin{figure}[htbp]
\begin{center}
\includegraphics[width=0.65\textwidth]{figs/install/jupyter_startup.png} 
\caption{Example starting Jupyter Notebook from the Linux command line.  In Windows, you will need to open the Anaconda Prompt instead of a terminal.}
\label{fig:jupyterstartup}
\end{center}
\end{figure}

\begin{figure}[htbp]
\begin{center}
\includegraphics[width=0.65\textwidth]{figs/install/jupyter_window.png} 
\caption{The Hello World example Jupyter Notebook.}
\label{fig:jupyterwindow}
\end{center}
\end{figure}

\begin{figure}[htbp]
\begin{center}
\includegraphics[width=0.65\textwidth]{figs/install/jupyter_saved.png} 
\caption{Example showing the saved Jupyter notebook.  Notice that notebook file (ipynb) is not human readable on its own: it requires the Jupyter software to render it in a human readable form.}
\label{fig:jupytersaved}
\end{center}
\end{figure}

To activate the phy40 environment type:
\begin{lstlisting}{language=csh}
  conda activate phy40
\end{lstlisting}
When you are done with Phy 40 for the day you can deactivate this
environment (later) with:
\begin{lstlisting}
  conda deactivate
\end{lstlisting}
Launch jupyter notebook with:
\begin{lstlisting}
  jupyter notebook
\end{lstlisting}
This should start the Jupyter Notebook server and open a client in your web browser.
An example starting a Jupyter Notebook from Linux is shown in Fig.~\ref{fig:jupyterstartup}.

You should create one Jupyter Notebook per lab assignment, by choosing
the New (Python 3) option in your client.  Change the name of your
notebook to something that clearly identifies the lab.  Start each lab
with comments (starting with ``\#'' symbol) indicating the title of
the lab, then your name followed by your lab partners.  See the first
cell of Fig.~\ref{fig:jupyterwindow} for an example.  This first cell
is also a good place to issue the ipython ``magic function'':
\begin{verbatim}
%pylab inline
\end{verbatim}
which will setup the notebook for inline plots and load the numpy and matplotlib libraries for you.

Each assignment will consist of a number of steps, clearly numbered like this one, your first step:\\

\plot Print ``hello world'' using the python print command.\\

\noindent
To keep your notebook clear, label cells (such as this one) with a
comment for the assignment step number, as in the second cell of
Fig.~\ref{fig:jupyterwindow}.  You only need to label one cell if the
assignment is fullfilled across several cells.

Jupyter Notebook checkpoints your work automatically.  You should be
able to see your notebook saved in the working directory where you
started, as in Fig.~\ref{fig:jupytersaved}.  Notice that while the
notebook file is ASCII text, it is not a human readable format.  The
Jupyter software is needed to render the notebook in a human readable
way.  To make your grader's life easier, you will be submitting PDF
versions of your notebook, once all of the tasks are completed and the
output is visible.  There are several ways to make a PDF file from
your notebook, but the most reliable is to use the ``Print Preview''
option to view the notebook as a PDF file within your browser, then
use the print feature of your browser to print the page as a PDF file.
Try this now, and make sure you can create a legible PDF file, but do
not submit it to the course site, as you still have more to do.
Always keep your python notebook file (ipynb) even after you submit
the assignment.  If you have problems, you can reproduce a PDF file
from the notebook file, but it is tedious to reproduce your notebook
from PDF.  If you have problems producing the PDF file, you can submit
the ``ipynb'' file as a temporary work-around, but work with your TA
to sort out the problem as quickly as possible.

\section{Submitting your assignment}

Before submitting, take some time to clean up your assignments to
remove anything superfluous and place the exercises in the correct
order.  You can also add comments as needed to make your work clear.
You can use the Cell $\to$ All Output $\to$ Clear and Cell $\to$ Run
All commands to make sure that all your output is up to date with the
cell source.

When you are satisfied with your work, print the PDF file as described
earlier and submit {\bf both} the PDF file and notebook file to the
course website.











\chapter{Binary Numbers}

\section{Introduction}

At the heart of numerical analysis, naturally, you will find numbers.
In this lab, we will explore the basic data types in Python, with
particular emphasis on the computer representation of integers and
real numbers.  All modern programming languages do an admirable job of
hiding the limitations of the computer representations of these
mathematical concepts.  In this chapter, we will deliberately explore
their limitations.

\section{Preparation}

This lab will rely on the material from Section 1.2.2 of the
Scientific Python Lecture notes.\\

\plot Enter the following code into a cell and check the output:
\begin{python}
a = 121
print(type(a))
print(a)
\end{python}
Next, in the same cell, add a line at the end setting $a$ to a real value, \pyth{a = 1.34}, and
print the type and value again.  Check the output.  Next, set $a$ to Boolean value,
\pyth{a = False}, and print the type and value yet again. \\

In many languages, such as C and C++, variables are strictly typed:
you would have to decide at the start whether you want $a$ to be of
integer, float, or boolean type, and then you would not be able to
change to a different type later.  Python variables are references to
objects, which means they only point to memory locations that contain
objects with all of the data and functionality associated with that
object.  When you write $a = 121$ it is interpreted as ``set variable
$a$ to point to a location in memory that contains a class of type
integer with the value 121''.\\

\plot Enter the following code into a cell and check the output:
\begin{python}
a = 12
b = a
b = 5
print(a)
print(b)
\end{python}  
Why is the output ``12, 5'' instead of ``5, 5''?  When you write
\pyth{b = a}, the variable $b$ points to the same Integer class that
$a$ points to.  So when you write ``b = 5'' why doesn't the value of
$a$ change as well?  This code snippet shows that it does not.  The
reason is that Integers are {\em immutable} objects in python... their
values cannot be changed.  So when you write \pyth{b = 5} it is
interpreted as ``variable b points to a (new) Integer with value 5.''
The variable a continues to point to the Integer with value 12.  The
``is'' operator used like this:
\begin{python}
  print(a is b)
\end{python}
tells you if $a$ references the same object as $b$.  Add to compiles
of this line to your snippet in order to clarify the situation.  One
call should return True and the other False.\\

\plot If I set a variable $a$ to an integer value and I set $b$ to
the same integer value, do $a$ and $b$ refer to the same object, or to
two different objects with the same value?  Write a snippet of code
(three lines) to find out.\\

In this lab, it will be convenient to know how to calculate the absolute value of a number, which you can do with the python \pyth{abs} function:
\begin{python}
a = -1.5
b = abs(a)
print(b)
\end{python}


\section{Binary Representation of Integers}

Computer hardware is based on digital logic: the electrical voltage of
a signal is either high or low, which correspond to a mathematical
zero or one.  A digital clock is used to ensure that signals are only
sampled at particular times, when they are guaranteed not to be in
transition from a zero to one or vice versa.  The Arithmetic-Logic
Unit (ALU) uses digital logic gates (such as AND or OR) to perform
calculations.  For example, it is possible to build an adder that uses
only NAND gates.

Because digital signals have only two states (zero and one), the most
natural way to represet numbers in a computer is using the base two,
which we call binary.  In the familiar base ten, we have ten digits
(from 0 to 9) and the place value increases by a factor of 10 with
each digit moving toward the left.  In binary, we have only two digits
(0 and 1) and the place value increase by factors of two.  A single
digit in binary is referred to as a ``bit''.  You add columns quickly
when counting in binary: zero(0), one(1), two(10), three(11),
four(100), five(101), and so on.  For efficient operations, computers
often group eight bits together to form a byte.  Digital values are
therefore commonly represented in hexidecimal (base 16) where two
digits of hexidecimal describes one byte.  See Table~\ref{tbl:binary},
which you can produce for yourself in Python like this:
\begin{python}
print("dec: hex: bin:")
for d in range(16):
    print("{0:<2d}   0x{0:01x}   0b{0:04b}".format(d))
\end{python}
It is conventional to preprend binary numbers with ``0b'' and
hexadecimal with ``0x'' otherwise we wouldn't know whether a ``10''
represents ten, sixteen, or two!  Notice that the largest number that
can be written with $n$ bits is $2^n-1$.

\begin{table}
\begin{center}
  \caption{The numbers 0 to 15 in decimal, hexadecimal, and binary.}
  \label{tbl:binary}
\begin{tabular}{|lll|lll|}
\hline
dec: & hex: & bin: & dec: & hex: & bin: \\
\hline
0  & 0x0  & 0b0000 & 8  & 0x8  & 0b1000 \\ 
1  & 0x1  & 0b0001 & 9  & 0x9  & 0b1001 \\
2  & 0x2  & 0b0010 & 10 & 0xa  & 0b1010 \\
3  & 0x3  & 0b0011 & 11 & 0xb  & 0b1011 \\
4  & 0x4  & 0b0100 & 12 & 0xc  & 0b1100 \\ 
5  & 0x5  & 0b0101 & 13 & 0xd  & 0b1101 \\
6  & 0x6  & 0b0110 & 14 & 0xe  & 0b1110 \\
7  & 0x7  & 0b0111 & 15 & 0xf  & 0b1111 \\
\hline
\end{tabular}
\end{center}
\end{table}

The mathematical notion of an integer can be naturally implemented by
computer hardware.  Although integers are represented in binary in the
hardware, modern compilers and languages generally print them to
screen as decimal by default.  One caveat is that computers do not
have an unlimited number of bits.  Many computer languages use 64-bit
integers, which means that only the integer values from 0 to
18446744073709551615 can be represented:
\begin{python}
x = 2**64-1
print(x) 
\end{python}
For signed integers, one bit is used to indicate the sign (positive or
negative) and so a 64-bit signed integer can represent integer values
from -9223372036854775808 to 9223372036854775807.  As long as an
integer value is within the range covered by the integer type, the
integer value can be perfectly represented.

Python uses arbitrary sized integers: it simply adds more bits as
needed to represent any number.  For an extremely large number, you
will eventually reach practical limitations on the amount of memory
and processing time available in the computer, which will limit how
large of an integer can be calculated.\\

\plot See for yourself just how huge integers can be in python by entering:
\begin{python}
x = 2**8000
print(x)
\end{python}
and checking the output.\\

\plot Print the integer 64206 in decimal, hexadecimal, and binary.  Hint: just reuse the carefully formatted print statement from the example above.\\

\plot Suppose you are tasked with rewriting the firmware for a 
distant satelite which has just lost one line from an eight-bit
digital communications bus due to radiation damage.  You now have only
seven working bits!  What is the maximum sized unsigned integer which
you could write on this degraded seven-bit bus?  What range of signed
integers could you write? \\

% FIXME:  in future years, use the paper and pencil macro here:
\plot This is a paper and pencil exercise.  You can do it on paper and
pencil and submit a scanned PDF, or you can just record you steps as
comments in a jupyter notebook cell.  Let's consider a four-bit signed
integer, so zero is 0000 and one is 0001.  Suppose the upper bit is
reserved for sign, so 1XXX is a negative number.  An obvious choice
for representing -1 would be 1001, but there is a better choice.
Consider that:
\begin{displaymath}
(-1) + 1 = 0  
\end{displaymath}
Well, if we simply ignore the last carry bit (5th bit):
\begin{displaymath}
1111 + 1 = 10000 = 0000  
\end{displaymath}
So if we define 1111 as -1, we can treat addition with negative numbers exactly the same as adding ordinary numbers.  Find the representation for -2 such that
\begin{displaymath}
-2 + 2 = 10000 = 0000
\end{displaymath}
then show that:
\begin{displaymath}
-1 + -1 = -2. 
\end{displaymath}
Python does not use this trick, but many other languages do.\\

\section{Binary Representation of Real Numbers}

Representing real numbers presents much more of challenge.  There are
an uncountably infinite number of real numbers between any two
distinct rational numbers, but a computer has only finite memory and
therefore a finite number of states.  It is impossible for computers
to exactly represent every real number.  Instead, computers represent
real numbers with an approximate floating point representation much
like we use for scientific notation,
\begin{displaymath}
x = m \times B^n
\end{displaymath}
where the significand $m$ is a real number with a finite number of
significant figures and the exponent $n$ is an integer.  The base $B$
is ten for scientific notation but typically two in a floating point
representation.  The exponent $n$ is typically chosen so that there is
only one digit before the decimal point in the base $B$, e.g. $3.173
\times 10^{-8}$ for scientific notation.

The limited precision of the discriminant can lead to challenges when
using floating point numbers.  The floating point precision is
specified by the parameter $\epsilon$ (epsilon) which is the
difference between one and the next highest number larger than one
that can be represented.  For scientific notation with four
significant digits, $\epsilon=0.001$, because we cannot represent
anything between $1.000 \times 10^0$ and $1.001 \times 10^0$ with only
four significant figures. \\

\plot Determine the Python floating point $\epsilon$ by running
\begin{python}
import sys
print(sys.float_info.epsilon)
\end{python}

\vskip 0.25cm
\begin{plot} \end{plot} Determine the Python floating point $\epsilon$ for yourself by running:
\begin{python}
eps = 1.0
while eps + 1 > 1:
    eps = eps / 2
eps = eps * 2
print(eps)
\end{python}
Here the while loop continues running the indented code until the
condition $\epsilon+1=\epsilon$ is met.

\newpage
\vskip 0.25cm
\plot Python uses the IEEE 754 double-precision
floating-point format.  This format uses 64-bits overall, with 52 bits
reserved for the significant.  The standard uses a clever trick to
save one bit, by requiring that the leading bit of the significant $m$ is
one, and defining 53 signficant figures using 52 bits:
\begin{displaymath}
m = 1.m_1 m_2 m_3 ... m_{52}
\end{displaymath}
Here each $m_i$ represents an individual bit (0 or 1).  Predict the
parameter $\epsilon$ and compare with the above. \\


\plot It seems like $\epsilon$ should be small enough to simply ignore
it in most cases, but in fact it shows up quite clearly if you apply
strict equality to floating point quanties.  To see the problem, run
this code, checking if $\sin(\pi)$ is zero:
\begin{python}
x = np.sin(np.pi)
print(x)
print(x==0)
\end{python}
The strict equality condition $x==0$ is not met because of limited
floating point precision.  Instead of strict equality, check that $x$
is near zero with a condition like:
\begin{displaymath}
|x| < 10 \epsilon
\end{displaymath}
where $\epsilon$ is the machine precision and the factor of 10 is a
conservative factor to allow for round-off errors that might be a bit
larger than the best possible precision $\epsilon$.  Add such a
condition to the code above and show that this looser definition of
equality now holds.  In general, when using approximations (like
floating point numbers) you can only check things to within the
accuracy of the approximation!\\

\plot Here is another case where floating point precision joins the
chat uninvited:
\begin{python}
x = 0.1
y = x+x+x
print(y == 0.3)
\end{python}
Devise an alternative to \pyth{y==0.3} that properly accounts for
floating point precision.\\

\plot Personally, I prefer my zero's to look like zero.  When floating
point limitations are making them look non-zero, I like to clean then
up with rounding, like this:
\begin{python}
x = np.sin(2*np.pi)
print(x)
x = np.around(x,15)
print(x)
\end{python}
Yeck! What is $-0$??!!  IEEE 754 defines two zeros $-0$ and $0$.  $-0$
is used to indicate that $0$ was reached by rounding a negative
number.  This is so that $1/-0$ can be interpreted as $-\infty$ and
$1/0$ as $+\infty$.  If you just want to make this go away, add ``+0'':
\begin{python}
x = np.around(x,15)+0
\end{python}
Show that this works.\\

\plot Consider the following code:
\begin{python}
a = 5;
b = 1.2343E-17;
sum = 0
sum += 5;
for i in range(1000000):
    sum = sum + b
print(sum)
\end{python}
Is there any problem here?  If there is, fix it by changing only the {\em order} of the lines of code.


\section{Other Data Types}

Integers and floating point numbers are the real work horses of
computational physics.  We'll add numpy arrays in a future lab.  This
section will briefly introduce the remaining types.

Python includes strings as a basic type:
\begin{python}
s = "hello world"
s = s + " (it's been a strange few years)"
print(s)
print(type(s))
print(s[6],s[4],s[6])
print(type(s[0]))
\end{python}
Strings are {\em immutable} objects that contain textual data.  If you
have used other languages, you might expect \pyth{s[0]} to be a
``character'' but in Python it is a string of length one.  There is no
built-in character type.

Python includes complex numbers as a basic type:
\begin{python}
  z = 1 + 2j
  print(type(z))
  print(z.real)
  print(z.imag)
  print(z.conjugate())
\end{python}
This is our first example of an object oriented programming (OOP)
class interface.  To compute the complex conjugate of $z$, an ordinary
function would need to be passed $z$ as an argument, or else it would
not know which complex value to use for the computation.  But
\pyth{z.conjugate()} is a {\em method} of the class complex.  The
method is tied to the instance of complex number $z$ by the ``.'' and
has access to all of the data it needs from $z$.  Similarly,
the \pyth{real} and \pyth{imag} are member data of class complex: they
are the integers that contain the real and imaginary parts of $z$.
Objects play a central role in Python, but in a refeshingly
understated and reserved manner.  It is enough for now to understand
that \pyth{z.conjugate()} is much like a function that already has $z$
as a parameter, and \pyth{z.imag} and \pyth{r.real} are just ways to
access the data contained in $z$.\\

\plot Define a complex number with value of $1+i$ and multiply it by
it's complex conjugate.  Show that the resulting complex number has
zero for it's imaginary part.\\

Python provides Lists as a native container of python objects.  We'll
make much more use of numpy arrays, which are better suited to
numerical analysis, but Python Lists occassionally play a role for various
bookkeeping tasks:
\begin{python}
L = ["hello", 1, 2, 3+2j, 3.45, "green"]
print(L[0])
print(L[1])
print(L[5])
L[0] = "goodbye"
print(L)
\end{python}
Here the List $L$ contains a variety of (admittedly rather useless)
objects.  These objects can be referred to individually by their
index.  One place where lists really shine is in looping over a
custom list of values:
\begin{python}
for i in [1,5,10,50,100]:
    print(i)
\end{python}

\vskip 0.25cm
\plot Run the following code:
\begin{python}
  a = [1,2,3,4,5]
  b = a
  b[0] = 3
  print(a[0])
  print(a is b)
\end{python}
(Spits out coffee) ``What the???!!''  Lists are {\em mutable} which makes
them fundamentally different from {\em immutable} integers.  Here
we assign $b$ to point to the same object as $a$ (a list) and then
change an entry in that list.  Even after the change, $a$ and $b$
point to the same object.  This is only possible because the list
object is mutable. \\

\plot It's good to read the documentation, but it's a useful skill to
figure things out for yourself too!  Without looking up the
documentation, write a snippet of code to determine for yourself if
complex numbers are mutable (like lists) or immutable (like integers).
We are able to change just a part of a list (like \pyth{a[0]=3}).  But can we change part of a complex number (like \pyth{z.real}).  Try it and find out!

(It's OK if your code throws an error here, but you can also comment
it out if you are the sort of person that can't possibly leave it
alone)\\

















\chapter{Sequences and Series}

\section{Introduction}

In this lab, we will apply for loops to study sequences and series.
If you already have programming experience, you can complete a
challenge problem in lieu of some of the other problems: see the final
problem for the details.

\section{Preparation}

This lab will rely on the material from Sections 1.2.1 to 1.2.4 of the
Scientific Python Lecture notes.  Most of the problems can be
completed using a simple functions containg a single for loop, such as
in this function:
\begin{python}
  def loop(n):
    for i in range(n):
        print(i)
\end{python}
To run the code in the function, you call the function, usually in a different cell:
\begin{python}
loop(5)
\end{python}

\vskip 0.25cm
\plot Create a new function:
\begin{python}
def mult(a,n):
   # your code here ...
\end{python}
that prints the first n multiples of a.  For example \pyth{mult(3,4)} should output:
\begin{python}
3
6
9
12
\end{python}
In future problems, we'll describe this output simply as 3, 6, 9, 12.
We won't be picky about whitespace unless we discuss it explicitly.
One way to complete this is to use the three arguments of
\pyth{range(start,stop,step)}.

\section{Fibonacci Sequence}

The Fibonacci numbers are a sequence of numbers satisfying the
recursion relationship:
\begin{displaymath}
F_{n+2} = F_{n} + F_{n+1}
\end{displaymath}
with $F_0=0$ and $F_1=1$.  The sequence is:
\begin{displaymath}
0, 1, 1, 2, 3, 5, 8, 13, 21, 34, \ldots
\end{displaymath}
This sequence can be generated numerically by an algorithm such as this one:
\begin{algorithm}
fa := 0 # set fa to 0
fb := 1 # set fb to 1
repeat n times: 
   fc := fa + fb
   print fc to screen
   fa := fb
   fb := fc
\end{algorithm}
Note that this is not python syntax.  What is the importance of the
last two lines?  Would the algorithm work if we exchanged their order?\\

\plot Use the algorithm described above to implement a new function
\pyth{fib(n)} which prints the next $n$ Fibbonacci numbers after the
initial 0 and 1.\\

\section{Arithmetic Series}

The finite arithmetic series
\begin{displaymath}
  S_n = \sum_{k=0}^{n} (a + kd) = a + (a+d) + (a+2d) + \ldots + (a+nd)
\end{displaymath}
sums to the average of the first and last terms times the number of terms:
\begin{equation} \label{eqn:arithsum}
S_n = (n+1)\frac{a + (a+nd)}{2}
\end{equation}
We will assume $a=d=1$ and calculate this finite series numerically using the following function:
\begin{python}
def arith(n):
     sum = 0
     for k in range(n):
         sum = sum + 1 + k 
         #print("k: ", k, "\t sum: ", sum)
    return sum    
\end{python}
Type in this function and see how it works by uncommenting the print
statement (delete the \# symbol that starts a comment) and calling it
as \pyth{arith(5)}.  The use of print statements in a loop like this
or at each stage of a calculation is a simple, effective and classic
debugging technique.  You test your code with the print statements
included, keeping n small so you don't fill your whole screen with
output. Once your code is working, you comment out the unneeded print
statements so that the interpreter ignores them and you no longer
see the unneeded output.  Why not just delete them?  You can, but
experience shows that if you do, you will need the line again shortly!\\

\plot Obtain the sum of the first $n$ terms of arithmetic series with
\pyth{sum = arith(n)} for three different values of $n$.  Each time,
show that sum returned by the function matches the expected sum.

\section{Geometric Series}
\label{sec:geom}

The geometric series
\begin{displaymath}
  \sum_{k=0}^{\infty} a r^k = a + ar + ar^2 + ar^3 + \ldots
\end{displaymath}
converges for $|r| < 1$ to:
\begin{equation} \label{eqn:geomsum}
  \frac{a}{1-r}.
\end{equation}
We will demonstrate this numerically.\\

\plot Implement a function \pyth{geom(a,r,n)} which calculates sum of the first $n$ terms of the geometric series with $k$th term $a r^k$.  Show that it agrees with Eqn.~\ref{eqn:geomsum} for $a=2$, $r=0.5$  $n=100$.
\\

\plot Call you geometric series function again for $a=3$, $r=0.8$ and $n=100$.  Compare with the expected output calculate within python and with pencil and paper.  Do they agree exactly?  If not, do they agree within the floating point precision?
\\

\plot Now compare your calculated sum with Eqn.~\ref{eqn:geomsum} for $a=1$, $r=-0.9$  $n=100$.  How is the agreement?  Increase $n$ and see what happens.  Why do you suppose this series is slower to converge?\\

\section{Refinements}

There are a few refinements you can make to your code.  Don't change
your working code from previous examples!  Instead, copy the previous
version to a new cell and make your refinements there.  You don't even
need to change the name of the function, Python will happily overwrite
the old function implementation when it reaches the cell with the new
version.  Make these code improvements:\\

\plot  Improve your Fibbonacci function so that prints the
first $n$ numbers including the initital two numbers ``0'' and ``1''.
Make sure it works properly for $n=0$, $n=1$, $n=2$, and so on.\\

\plot  Extend the Arithmetic series function to include
parameters $a$ and $d$.  Show that it works.\\

\section{Maclaurin series}
\label{sec:mac}
A Maclaurin series can be used to approximate a function by calculating its derivative close to the point of interest.

\begin{displaymath}
f(x) = f(0) + f'(0) x +\frac{f''(0)}{2!}x^{2} + \frac{f'''(0)}{3!}x^{3}...
\end{displaymath}

This is equivalent of approximate a function with a polynomial. Larger is the polynomial, better is the approximation even at points of the function far away from the referent point. 


Let's star defining a function \pyth{f(x) = cos(x)}.
For this exercise, we will use the numpy package that we will use extensively later in the course. 

\begin{python}
import numpy as np
def myfunc(x):
    return np.cos(x)    
\end{python}

Let's now define a second order polynomial that approximate this function (around 0) and let's write a python function with this polynomial. To find this function you compute the derivatives of f(x) and can use the Maclaurin series:
\begin{displaymath}
P(x) = 1 - \frac{1}{2!}x^{2} 
\end{displaymath}

\begin{python}
def polynomialorder2(x):
    return  1 - 1/2. * x**2 
\end{python}


\plot Compute the derivative of the function f(x) around the point 0 and define the polynomial that approximate the function f(x) with order 4, 6 and 8. Define a new function for each of these polynomials. \\


\plot Investigate how good is your polynomial or second order to approximate f(x) = cos(x) when you start to go far away from x=0. In particular, fix a value for x (x = 1) and compare in absolute value the difference between  
\pyth{f(x) - polynomialorder2(x)}  

\begin{python}
x = 1 
print(np.abs(myfun(x) - polynomialorder2(x)))
\end{python}

\plot  Investigate how good are the polynomial of orders 2 (P2), 4 (P4), 6 (P6) and 8 (P8) to approximate the function f(x). Find how far can you go with x and still have difference between f(x) and your polynomials (P2,P4,P6,P8) $<$ than .1. 


\section{Fibonacci Integer Right Triangles}

Starting with the number 5, every second Fibonacci number is the
length of the hypotenuse of a right triangle with integer sides.  The
first two are:
\begin{displaymath}
5^2 = 3^2 + 4^2  
\end{displaymath}
and 
\begin{displaymath}
13^2 = 5^2 + 12^2.  
\end{displaymath}
Furthermore, from the second triangle onward, the middle side is the sum of the lengths of the sides of the previous triangle, for example:
\begin{displaymath}
12 = 3 + 4 + 5.  
\end{displaymath}

\vskip 0.25cm
\plot (Optional Challenge) Use numerical methods to
explicitly verify these properties for the first $n$ Fibonacci integer
right triangles.\\

If you would prefer, you may submit the Optional Challenge problem plus the
problems from Section~\ref{sec:geom} and Section \ref{sec:mac} to complete the assignment.


\chapter{The Quadratic Equation and Prime Numbers}

\section{Introduction}

In this lab, we will make more extensive use of conditional statements
to implement algorithms which solve the quadratic equation, identify
prime numbers, and add fractions.  An optional challenge problem,
The Lucky Number of Euler, explores how the quadratic equation can
generate prime numbers.

\section{Preparation}

This lab will rely on the material from Sections 1.2.1 to 1.2.4 of the
Scientific Python Lecture notes.  We'll now be making frequent use of
conditional statements:
\begin{python}
def compare(a,b):
    if (a==b):
        print("a equals b")
    elif (a<b):
        print ("a is less than b")
    else:
        print ("a is greater than b")
\end{python}
Notice that Python uses \pyth{==} for comparison.  You will get a
syntax error if you use \pyth{a=b} instead.

We'll also use of the modulo operator $\%$ extensively.  The modulo
operation $a\%b$ returns the remainder from the integer division
$a/b$.
\begin{python}
# is b a factor of a?
def isfactor(a,b):
    if (a%b == 0):
        return True;
    return False
\end{python}
Why does \pyth{a%b == 0} mean that b is a factor of a?\\

\newpage
\vskip 0.25cm
\plot Consider this verbose code snippet:
\begin{python}
for i in range(100):
    print("on index ", i)
\end{python}
Which prints the current index on every iteration.  Use the modulo
operator to modify the code so that it only prints the index every 10
iterations.  This is a classic trick!\\

We will also be using while loops, which repeat a block of code until a condition is met:
\begin{python}
count=0
while(count<10):
    print(count)
    count = count+1
\end{python}

\section{Quadratic Formula}

The Quadratic equation:
\begin{displaymath}
  ax^2 + bx + c = 0
\end{displaymath}
has solutions which are given by the quadratic formula
\begin{equation} \label{eqn:quad}
x = \frac{-b \pm \sqrt{b^2 - 4ac}}{2a}
\end{equation}
The number of unique real solutions depends on the quantity in the
square root, which is called the discriminant:
\begin{displaymath}
b^2-4ac
\end{displaymath}
If this is positive there are two real solutions, it it is one there
is one real solution, and if it is negative there are no real
solutions.  For now, let's assume that a, b, and c are all integers.

In this case, the solution to the quadratic equation can be calculated
as follows:
\vskip -0.25cm
\noindent
\begin{algorithm}
  D := $b^2 - 4ac$
  if (D=0):
     calculate the single solution from quadratic 
     print single solution
  if (D<0): 
     print no solutions
  if (D>0):
     calculate both solutions from quadratic formula
     print both solution
\end{algorithm}
\vskip 0.25cm

A test case with one real solution is:
\begin{displaymath}
(x-1)(x-1)= x^2 -2x + 1.
\end{displaymath}
A test case with two real solution is:
\begin{displaymath}
(x-1)(x+1)= x^2 - 1.
\end{displaymath}
A test case with zero real solutions is:
\begin{displaymath}
(x-i)(x+i)= x^2 + 1.
\end{displaymath}

\vskip 0.25cm
\plot Implement a function \pyth{quad(a,b,c)} which reports the solutions to the quadratic equation and verify it with the test cases shown.\\

\vskip 0.25cm
\plot Calculate three more test cases with integer solutions and use them to test your function more thoroughly. \\

\vskip 0.25cm
\plot (Optional) The condition that the discriminant is exactly zero
(\pyth{D==0}) is problematic when applied to floating point
numbers. (Why?) In example is for a=.1 b=.3 c =0.225 which should have
only one real solution.  Test you function with this test case.
Modify the conditionals in your function to account for floating point
precision and test it.\\

\section{Prime Numbers}

A prime number is a number that has two and only two factors: itself
and one.  One is not prime, but two is.  We can determine if a number
$a$ is prime as follows:
\vskip 0.25cm
\begin{algorithm}
  if (a<2):
     return false
  i := 2
  while ($i \leq \sqrt{a}$):   
     if (a%i=0):
        return false
     i := i + 1
  return true
\end{algorithm}
\vskip 0.25cm
Why is there no need to check for factors larger than $\sqrt{a}$?\\

\plot Implement a function \pyth{isprime(a)} which returns \pyth{True}
if the integer $a$ is prime and \pyth{False} if not.\\

Suppose that next we want to find the first $n$ prime numbers greater
than or equal to a number $A$.  We can simply check if $A$, $A+1$,
$A+2$, and so on are prime until we find the first $n$.  But we do not
know how many numbers we will have to check before finding $n$ that
are prime.  This is a case for a \pyth{while} loop.\\

\plot Find the first $n$ prime numbers greater than $A$ using a while
loop and your \pyth{isprime} function. Test it for $n=10$ and $A=0$,
then for $A=1000000000$.  Try that with paper and pencil!\\

Notice how we broke this problem of finding primes into two parts:
determining whether or not a number is prime or not, then testing and
counting prime numbers.  We thoroughly tested the first part before
using it in the second.  This is an essential approach to solving
computational problems: break complicated tasks down into smaller
tasks which can be tested separately.\\

\plot Implement a function which computes the fraction
\begin{displaymath}
\frac{n}{d} = \frac{a}{b} + \frac{c}{d}
\end{displaymath}
from integer inputs $a$,$b$,$c$ and $d$ and returns integers $n$ and
$d$.  Returning $n>d$ is allowed, but $n/d$ should be a simplified
fraction with a greatest common factor of one.  You can return two
integers as a Tuple, like this:
\begin{python}
  # function which	adds fractions
  def addfrac(a,b,c,d):
     n=d=0
     #your code...
     return n, d
     
  # calling function:
  n,d =addfrac(1,2,1,3)
  print("{0}/{1}".format(n,d))
\end{python}
For full credit, you must factorize (see what I did there?) this
problem into two parts: your \pyth{addfrac} funtion should call a
second function that does one well defined task.
\section{The Lucky Number of Euler}

Euler discovered that the formula:
\begin{displaymath}
k^2 + k + 41
\end{displaymath}
produces prime numbers for $0 \leq k \leq 39$.  Perhaps you can beat Euler at his own game!

Consider quadratics of the form:
\begin{displaymath}
k^2 + ak + b
\end{displaymath}
For each integer value of $a$ and $b$, there is a maximum number $n$ such that the quadratic formula produces prime numbers for $0 \leq k < n$\\

\plot (Optional) Find the the values of $a$ and $b$ which produce the largest number of prime numbers.  Restrict yourself to $|a| \leq 1000$ and $|b| \leq 1000$.\\ \vskip 1cm

\noindent
If you do complete this optional problem, then nice work, hot shot,
but remember that Euler found his without using a computer!



\chapter{Arrays, Plotting and Chaos}

\section{Introduction}

This lab will introduce a fundamental element of scientific python,
the numpy array, and use them to produce plots.  We will also consider
the non-linear logistics map and examine bifurcation in the approach
to chaos.

If you can complete all of the plots in Section~\ref{sec:logmap}
including the optional plot and {\em without any instructor help} then
you may omit the plots from the preceding sections.

\section{Preparation}
\label{sec:arraysprep}

This lab will rely on the material from Sections 1.4.1 to 1.4.2 and
1.5.1 to 1.5.2 of the Scientific Python Lecture notes.  This is the
first lab that relies on inline plotting, so make sure you are
starting your notebook with the ``line magic'':
\begin{python}
  %pylab inline
\end{python}
This will load the numpy library as np, the matplotlib.pyplot library
as plt, and setup the matplotlib backend to imbed plots in your
notebook.

A Numpy array is a grid of values.  Unlike Python lists, the elements
of a numpy array all have the same data type, which makes them much
more computionally efficient.  Choices for the data type include the
built-in python integer, float, and bool types.  The numpy library
provides a wide range of analysis tools that are mostly centered on
the numpy array type.

Numpy arrays can be constructed easily from a Python list:
\begin{python}
a = np.array([1.3,7.2,4.1,0.0])
b = np.array([[1,2],[3,4]])
print(a)
print(b)
print(np.shape(a))
print(np.shape(b))
\end{python}
This is conveninent when you have specific values you want to define by hand.  Another possibility is to construct the numpy array by calling a function designed specifically for the purpose:
\begin{python}
a = np.linspace(0,1,11)
print(a)
b = np.arange(0,5,1)
print(b)
\end{python}
Both \pyth{linspace} and \pyth{arange} allow you to specify the range
of values you want, but with \pyth{linspace} you specify the number of
points you want whereas with \pyth{arange} you specify the step size.

One of the great joys of using numpy arrays comes from the fact that
most operators are applied elementwise automatically, without the need
to explictly write a for loop:
\begin{python}
a = np.arange(0,5,1)
b = 10
print(a)
print(b)
print(a+b)
\end{python}
Notice how the value of $b$ (10) is added to {\em every element} of
$a$, without the need to explicitly loop over every element.  Try
modify the example code to multiply every element of $a$ by $b$. Then
try raising each element of $a$ to the power of 2. \\

\plot Use numpy arange and elementwise operations to implement a
function \pyth{def powers(a, n)} which returns a numpy array
containing the first powers of $a$ from 1 to $a^n$.  So for example
\pyth{print(powers(2,4)} should output \pyth{[ 1 2 4 8 16]} \\ vskip
1cm

Numpy arrays can also be built up element by element using the append
function:
\begin{python}
a = np.array([])
print(a)
a = np.append(a, 1)
print(a)
a = np.append(a, 3)
print(a)
a = np.append(a, 4)
print(a)
\end{python}
This example creates an empty numpy array and then adds one element at a time.\\

\plot Run the snippet above and observe the output.\\

\plot Implement a function \pyth{def recur(n,x)} which returns an
array containining the first $n$ values of the recurrance relationship
$x_{i+1}=2 \, x_{i}+1$ starting from $x_0=x$.  Use the following algorithm:
\begin{algorithm}
  Parameter $x$ # starting value
  Parameter $n$ # number of iterations
  Create empty array $a$ 
  Repeat $n$ times:
     append $x$ to array $a$
     $x$ := $2\,x + 1$
  print $a$
\end{algorithm}
Test your code for $x=1$ and $n=5$ and ensure that it produces the
correct output: \pyth{[ 1.  3.  7. 15. 31.]}.\\

\section{Plotting with Scentific Python}

Basic plotting in Python requires two numpy arrives: one for the $x$
coordinates and one for the $y$ coordinates.  Consider the following
very simple plot:
\begin{python}
x = np.array([0.0, 1.0, 2.0, 3.0, 4.0,  5.0])
y = np.array([0.3, 3.2, 5.8, 9.0, 12.4, 14.7])
plot(x,y,"bo")
\end{python}
Here, the ``bo'' options specifies blue circles.  Now consider:
\begin{python}
x = np.linspace(0, 1, 100)
y = np.sin(np.pi*x)
plt.plot(x,y,"r-")
\end{python}
Here the ``r-'' option specifies red line.  Including 100 points (as
done here) results produces a smooth looking curve.

Now promise me that you will never make another plot without labeling
the $x$ and $y$ axes! Here's another example will all the bells and
whistles you need to make a professional looking plot:
\begin{python}
UPPER = 2
LOWER = 0
tau   = 2*np.pi
x = np.linspace(LOWER,UPPER,100)
s = np.sin(tau*x)
c = np.cos(tau*x)
plt.plot(x,s,"b-",label="sin")
plt.plot(x,c,"r-",label="cos")
plt.xlabel("x")
plt.ylabel("y")
plt.title("Two Periods of a Sine and Cosine")
plt.legend(frameon=False)
plt.show()
\end{python}
Make sure you understand all of the features demonstrated here:
\begin{itemize}
 \item Variables \pyth{UPPER} and \pyth{LOWER} located at the top of
   the snippet, allowing for easy adjustment of parameters that affect
   the plot.
 \item Use of \pyth{np.linspace} to define an array of x values, with
   plenty of them (100) to produce nice smooth curves.
 \item Creation of two different arrays of y values, one for sin and one for cos.
 \item Plotting the arrays of $x$ and $y$ values with \pyth{plt.plot} using the ``-'' option for a line and color blue(``b'') for sin and red(``r'') for cos. 
 \item Defining appropriate axis labels with \pyth{plt.xlabel} and \pyth{plt.ylabel}. 
 \item Adding a title with \pyth{plt.title}
 \item Creation of a legend using the {\tt label} optional argument to {\tt plt.plot} and the {plot.legend()} command.  Removing the frame with option \pyth{frameon=False}
\end{itemize}
It is written so concisely and intuitively, you might not even notice
what is going on with the line:
\begin{python}
s = np.sin(tau*x)  
\end{python}
Remember that $x$ here is a numpy array of 100 elements.  The
\pyth{tau*x} multiples every element of x by our value tau.  The
\pyth{np.sin(tau*x)} then takes the sine of each element.  The
resulting numpy array, also of 100 elements, is referenced by variable
s.  Each element of $s$ contains $\sin(\tau x)$ for the corresponding
element of the array $x$.  It takes some getting used to for
programmers used to explicitly writing for loops for things like this,
but ultimately, the fact that python handles so much of this
bookkeeping for us is what makes it a very fun language to work with.\\

\plot Plot the $sinc(x)$ function as a smooth line in the $x$ range from -5
to 5.  Add appropriate labels to each axes.  Include a legend identifying the
sinc function.  For the line color, use any color other than red or
blue.

\section{The Logistics Map}
\label{sec:logmap}

The logistics map is the recurrence relation
\begin{displaymath}
p_{i+1} = r \; p_i \; (1 - p_i)
\end{displaymath}
which calculates the value of $p$ for step $i+1$ from step $i$.  The
variable $p$ can represent the ratio of a population to its maximum
possible value, and each iteration ($n$) is a step in time (such as
one year).  Each year, the population increases due to birth and
decreases due to starvation as the population approaches it's maximum
value ($p$ near 1).  The growth (or decline) of the population is
controlled by the growth rate parameter $r$.  We will only consider
$r$ in the range from $[0,4]$ which keeps $p$ in the range $[0,1]$.
This is a simple non-linear model which illustrates chaotic behavior.\\

\plot Implement a function \pyth{def logmap(p, r)} which returns the
next iteration ($p_{i+1}$) of the logistic map for parameter $r$ and
$p_i = p$.  Test your code by showing that for $r=3.0$ and $p_0=0.1$
the next five iterations are: 0.27, 0.5913, 0.725, 0.5981 and 0.7211.\\

\plot Use your \pyth{logmap} function to build an array containing the
first $n$ entries in the logistics map starting from $p_0=0.1$.  Check
that the output is correct by comparing with the results from the
previous exercise for $n=6$.\\

\begin{figure}[htbp]
\begin{center}
\includegraphics[width=0.65\textwidth]{figs/plotting/converge.pdf} 
\caption{Convergence of the logistic map for $r=2.8$}
\label{fig:logmapconv}
\end{center}
\end{figure}

\plot Plot the time evolution of the logistics map for $r=2.8$ for 40
iterations, starting from $p_0=0.1$ as in Fig.~\ref{fig:logmapconv}.
To create a plot, you'll need an array containing the $p$ values
(these correspond to the $y$ axis in the plot) which you can construct
as in the previous exercise.  But what about the $x$ axis values?
Since your array of $p$ values contains $[p_0, p_1, p_2 \ldots p_n]$
the corresponding array of indices is simply $[0,1,2 \ldots 3]$ which
you can construct using \pyth{np.arrange}.\\

Your results from the previous exercise should reproduce
Fig.~\ref{fig:logmapconv}.  This shows that for $r=2.8$ the logistics
map converges to a value of about 0.64.  But this non-linear recursion
relationship does not always converge to a single value.\\

\plot Plot the time evolution of the logistics map for $r=3.2$ for 40
iterations, starting from $p_0=0.1$.\\

Notice that now the system oscillates between two values (near 0.5 and 0.8).\\

\plot Plot the time evolution of the logistics map for $r=3.5$ for 40
iterations, starting from $p_0=0.1$.\\

\begin{figure}[htbp]
\begin{center}
\includegraphics[width=0.65\textwidth]{figs/plotting/logmap.png} 
\caption{Long-term behavior of the logistics map as a function of parameter $r$.}
\label{fig:logmap}
\end{center}
\end{figure}

Notice that now the system oscillates between four values.  This is an
example of bifurcation in the approach to chaos.  To show this more
clearly, we'd like to produce a plot as in Fig.~\ref{fig:logmap}.  For
each $r$ value, this shows the $p$ values from 100 iterations {\em
  after} the first 1000 iterations.  This shows the {\em long term
  behavior} of the logistics map.  We can clearly see that for $r=2.8$
the $p$ values converge to one single value and for $r=3.2$ there are
two values just as in the previous exercises.  These bifurcations
continue until the system becomes chaotic (oscillating between many
different values) with occassional windows of stability.

To produce this plot for yourself, start by considering this snippet:
\begin{python}
Nr = 5
r = np.linspace(2.6,4,Nr)
p = logmap(0.1,r)
print(p)
p = logmap(p,r)
print(p)
\end{python}
Here we create a numpy array of $r$ values, and pass that to our
\pyth{logmap} function instead of a single value.  The operations
within the function are applied elementwise to the array, and the
result is that instead of a single $p$ value, the call to to
\pyth{logmap(0.1,r)} returns an array of $p$ values, one for each $r$
value.  This is just what we need to make the plot in Fig.~\ref{fig:logmap}.\\

\plot Run (and understand!) the snippet and make a plot of $p$ versus
$r$.  Understand that you are plotting $p_2$ as a function $r$!  Use
the \pyth{\"k,\"} format option (black pixels) when plotting.  Comment
out the print statements and increase $Nr$ to 100.\\

\plot Make a plot that is {\em almost} like that of
Fig.~\ref{fig:logmap} by plotting $p_{1001}$ versus $r$.  Hint:
in the code above, instead of one call to \pyth{p = logmap(p, r)} use a for loop which
calls this 1000 times.\\

You should start to see features of the Fig.~\ref{fig:logmap} but you are only plotting one $p$ value for each $r$.  To see the bifurcations and chaos, you'll need to plot about 100 $p$ values at each $r$ value.\\

\plot Reproduce the plot in Fig.~\ref{fig:logmap}.  Hint: instead of plotting just $p_{1001}$ as in the previous exercise, plot the next 100 values as well.  Increase the number of $r$ values plotted to 1000.  Add appropriate labels to each axis.\\

\plot (Optional) The bifurcation diagram which you have constructed
exhibits self-similarity.  If you zoom into an appropriate region of
the diagram, you will find a diagram which looks quite similar to the
original diagram.  Produce a plot that demonstrates this
self-similiarity by zooming into a particular region.


\chapter{Differentiation and Projectile Motion}

\section{Introduction}

In this lab we will apply numerical differentiation to elementary
functions.  We'll apply the Euler method to compute the trajectory of
a projectile, and compare our results with the analytic solution.
We'll also show that our numerical technique easily accommodates air
resistance, a problem that has no analytic solution using elementary
functions.

\section{Preparation}

Our code is going to get complicated enough that we will need to pay
some attention to variable scope, as illustrated here:
\begin{python}
i=1
j=2
k=3
def f(i,j):
    print(i,j,k)
f(i,j)
f(j,i)
\end{python}
Try to predict the output of this snippet before running it.  The
first three lines define integers $i$,$j$, and $k$. These have global
scope, which means they can be accessed from anywhere.  The function
\pyth{f(i,j)} has parameters $i$ and $j$ which have a scope limited to
the function $f$.  Even though they have the same name as the global
variables $i$ and $j$, they are independent quantities.  Within the
function $f$ the variable $i$ is the first parameter, and $j$ is the
second parameter.  Because they have the same name, the global
variables $i$ and $j$ are {\em shadowed} by the local parameters $i$
and $j$.  The global variable $k$ is not shadowed.

In this lab, we will also be passing a function as an argument to another
function, as in this simple example:
\begin{python}
def show(f):
    print(f(2))
    print(f(3))    
    
def f(i):
    return i**2

def g(i):
    return i**3

show(f)
show(g)    
\end{python}
Here the \pyth{show} function takes another function \pyth{f} as an
argument.  Within the show function, the function \pyth{f} is called
using paranthesis just like any other function, as in \pyth{f(2)} We
define two additional functions, \pyth{f} which returns the square of
its argument, and \pyth{g} which returns the cube.  When
\pyth{show(f)} the output 4 and 9.  When \pyth{show(g)} the output 8
and 27.  Run the code as is, and also check what happens if you define
\pyth{g(i)} to require a second argument as in \pyth{g(i,j)}.

\section{Numerical Differentiation}

In lecture we derived the right derivative (aka foward derivative) formula
\begin{displaymath}
  f'(x) = \frac{f(x+h) - f(x)}{h} + \mathcal{O}(h)
\end{displaymath}
for numerically determining the derivative of the function $f$.  Remember we do not calculate the 
$\mathcal{O}(h)$ term, that indicates that the truncation error is of order $h$.
We also derived the centered derivative formula:
\begin{displaymath}
f'(x) = \frac{f(x+h) - f(x-h)}{2h} + \mathcal{O}(h^2)
\end{displaymath}
To evaluate a derivative using any of these formulas, we need to
choose an appropriate value of $h$.  If $h$ is too large, the
truncation error (the amount the estimated value differs from the
actual value) will dominate.  If $h$ is too small, we will encounter
problems with floating point precision.\\

\plot Implement the right derivative formula as \pyth{right(f, x, h)}
where $f$ is the function to be evaluated, $x$ is the location to
evaluate the derivative, and $h$ is the step size for the numerical
integration.  Check you code on several functions with known
derivatives, like this:
\begin{python}
def f(x):  # derivative 0
    return 2
def g(x):  # derivative 3
    return 3*x
def h(x):  # derivative 4x
    return 2*x**2

print(right(f,1,0.01))
print(right(g,1,0.01))
print(right(h,1,0.01))
\end{python} \vskip 0.25cm


\plot Implement the center derivative function as \pyth{center} and test it in the same manner as in the previous exercise for \pyth{right}.\\

\newpage

\plot Compare the performance of \pyth{right} and \pyth{center} like this:
\begin{python}
def f(x):  # derivative 6x**2
    return 2*x**3

print("right:", right(f,1,0.1),   "center:", center(f,1,0.1))
print("right:", right(f,1,0.01),  "center:", center(f,1,0.01)) 
print("right:", right(f,1,0.001), "center:", center(f,1,0.001)) 
\end{python}
Recall that the truncation error goes as $h$ for the right derivative and as $h^2$ for the center derivative.  Are these results consistent with that expecation?\\


From now on, we will use the center derivative function only due to
its better performance.  We can plot the derivative of a function like
this:
\begin{python}
def f(x):
    return 0.5*x**2

x = np.linspace(0,1,100)
y = center(f, x, 0.1)
plt.plot(x,ya,"-b")
plt.xlabel("x")
plt.ylabel("y")
plt.show()
\end{python}
Notice how the argument $x$ passed to the function \pyth{right(f,x,h)}
and then to \pyth{f(x)} is now a numpy array of 100 values from 0 to
1.  The derivative is now evalued at 100 places with a single call.\\

\plot Define $f(x) = x^3$.  Use your \pyth{center} function to evaluate it's derivative $f'(x)$ in the x range $[-2,2]$.  Plot both $f(x)$ and $f'(x)$ in that range (in the same plot) using different colors for each.  Add a legend and axis labels.\\

\plot Define $f(x) = \sin(x)$ using the \pyth{np.sin} function.  Use
your \pyth{center} function to evaluate it's derivative $f'(x)$ in the
x range $[0,2\pi]$.  Plot $f(x)$, $f'(x)$, and $\cos(x)$ in that range
(all in the same plot) using different colors for each.  Add a legend
and axis labels.\\

\section{Projectile Motion}

In lecture, we derived the Euler Method for iteratively determining the trajectory of a particle:
\begin{eqnarray*}
  \vec{v}_{n+1} &=& \vec{v}_n + \tau \vec{a}_n \\
  \vec{r}_{n+1} &=& \vec{r}_n + \tau \vec{v}_n \\
\end{eqnarray*}

\plot  Implement a function 
\begin{python}
def euler(dt, x, y, vx, vy, ax, ay):
    # your code here
    return x, y, vx, vy
\end{python}
which calculates the next iteration of $x$,$y$,$v_x$, and $v_y$ from
the current values of $x$,$y$,$v_x$,$v_y$,$a_x$ and $a_y$.  Notice that this function returns several different variables at once using a comma separated list referred to as tuple in python.  To retrieve the individual variables from the tuple, simply call the function like this:
\begin{python}
x,y,vx,vy = euler(x,y,vx,vy,ax,ay)
\end{python}
One downside of this convenient approach is that you must get the order of the variables correct!
Check you implementation against the following test values:
\begin{python}
print(np.around(euler(0.134, 0.659, 0.282, 0.662, 0.643, 0.900, 0.451),2))
print(np.around(euler(0.924, 0.959, 0.575, 0.299, 0.710, 0.699, 0.471),2))
\end{python}
and ensure that you get the correct output:
\begin{verbatim}
[0.75 0.37 0.78 0.7 ]
[1.24 1.23 0.94 1.15]
\end{verbatim}

We will use the Euler Method to simulate projectile motion.  We'll
take the initial velocity to be $20~\rm m/s$ and take $g=9.8~\rm
m/s^2$.  Here's a snippet of code that sets these constants and
determines the $x$ and $y$ coordinates of the initial velocity from an
angle $\theta$, which is set to $45^\circ$:
\begin{python}
tau = 2*np.pi
vi    = 20   # [m/s]
g     = 9.8  # [m/s^2]
theta = tau/8
dt = 0.01 # [s] 
x  = 0    # [m]
y  = 0    # [m]
vx = vi*np.cos(theta)
vy = vi*np.sin(theta)
\end{python}
The trajectory of the particle can be determined using the following algorithm:
\begin{algorithm}
  Create an empty array tjx # will contain $x$ positions of the trajectory
  Create an empty array tjy # will contain $y$ positions of the trajectory
  while $y \geq 0$:
     Append the $x$ position to tjx
     Append the $y$ position to tjy
     Compute the next values of $x$,$y$,$v_x$ and $v_y$ using the Euler Method
  Plot tjy versus tjx
\end{algorithm} 
Notice that the algorithm stops just before the projectile reaches $y \leq 0$.\\

\plot Use the Euler Method to plot the trajectory of a projectile with the initial conditions described above.\\

\plot Derive an expression (paper and pencil) for the maximum range of the trajectory and evaluate the range for these initial conditions.  Are the results consistent?\\

\plot Extend your simulation to record $v_x$ and $v_y$ at each step
along with the $x$ and $y$ positions.  Take the mass of the projectile
to be $m=0.145~\rm kg$ and plot the kinetic energy, potential energy,
and total energy as a function of time.  To build an array containing
the time of each step, for plotting quanties versus time, you can do:
\begin{verbatim}
t = np.arange(tjx.size)*dt
\end{verbatim}
Include a legend. The $x$ and $y$ axes have changed to time and energy, so make certain to change the axes labels!\\

\section{Projectile Motion with Drag}

In this section we will consider the effect of air restistance on a
baseball thrown a $20~\rm m/s$.  We can model drag as a deceleration:
\begin{displaymath}
\vec{a} = -k |\vec{v}| \vec{v}
\end{displaymath}
where $k=0.00622~\rm m^-1$ for typical baseballs.\\

\plot Extend your simulation to include the effect of drag.  Plot the
trajectory without drag and the trajectory with drag in the same plot.
Include a legend and (as always) label all axes.\\

\plot Including the effect of air restistance, plot the kinetic
energy, potential energy, and total energy (kinetic plus potential) of
the projectile as a function of time.  Is the total energy of the
projectile conserved?\\



\chapter{Pendulum Motion}

\section{Introduction}

In this lab we will apply the Euler and Verlet methods to the pendulum
problem.  We will compare the results of the Verlet problem to the
small angle approximation.

If you prefer, you can complete the shorter, but more challenging
sequence of problems: 7.8, 7.10, and at least two of the optional
challenge problems.  You will only receive minimal instructor help
while you are taking this approach.

\section{Preparation}

Suppose you want to produce a plot of $f(t) = A \exp(-t/\lambda)$
versus $t$ from $t=0$ to $10$ and $A=\lambda=5$.  For a visually
smooth plot, you want about $N=100$ points, but we'll start with a
smaller number $N=11$ for easy debugging.  You create an array containing the 11
time values you want to plot:
\begin{python}
MAX = 10
N = 11
t = np.linspace(0,MAX,N)
print(t)
\end{python}
You calculate the $y = f(t)$ values like this:
\begin{python}
A    = 5
LAMB = 5
y = A*np.exp(-t/LAMB)  # !!!
print(y)
\end{python}
Make sure that you thoroughly understand the line marked ``!!!''.  That is
calculating one $y$ value for every $t$ value, and so $y$ is an array
with the same shape and size as $t$:
\begin{python}
print("t shape: ", np.shape(t), "y shape: ", np.shape(y))
print("t size:  ", np.size(t),  "y size: ", np.size(y))
\end{python}
With two arrays of the same shape, plotting them is a simple matter.  Here we use the red line format, and add some axes labels:
\begin{python}
plt.plot(t,y,"r-")
plt.xlabel("t (s)")
plt.ylabel("y")
\end{python}
If you look closely, you'll see kinks in the plot for $N=11$.  Increase to $N=100$ for a visually smooth plot.

You are expected to do all of this on your own, from a prompt like this:\\

\plot Plot the function $f(t) = A \exp(-t/\lambda)$ from $t=0$ to 10, $A=5$ and $\lambda=5$ as a red line. \\

Now suppose instead that you already have a set of $y$-values in a \pyth{np.array} \pyth{yarr}, and you would like to plot $y$ vs $t$, knowing that these $y$ values were sampled starting at $t=0$ with a uniform step size of $\tau=0.2$ between each sample, i.e. at $t=0, 0.2, 0.4, \ldots$.  In this case, you would create a \pyth{np.array} containing your time values as:
\begin{python}
tau = 0.2
t = np.arange(yarr.size)*tau
\end{python} 
\vskip 0.25cm

\plot Starting from the $y$ values contained in:
\begin{python}
yarr = np.array([1,1,1,2,4,7,3,2,1,1,0,0])
\end{python}
which you know correspond to time values starting at $t=0$ with constant step size $\tau=0.5$.  Plot $y$ vs $t$ as blue circles and add axes labels.

\section{Pendulum Motion}

In lecture we showed a pendulum of length $L$ can be described by the angle $\theta$ with respect to vertical (rest position), the angular velocity:
\begin{displaymath}
\omega = \frac{d\theta}{dt}
\end{displaymath}
and the angular acceleration:
\begin{displaymath}
\alpha = \frac{d\omega}{dt} = \frac{d^2\theta}{dt^2} 
\end{displaymath}
In a constant gravitational field with acceleration $g$, the angular acceleration is:
\begin{displaymath}
\alpha = -\frac{g}{L} \, \sin \theta
\end{displaymath}
Which we can write as:
\begin{displaymath}
\alpha = - \omega_L^2 \, \sin \theta
\end{displaymath}
where
\begin{displaymath}
\omega_L = \sqrt{\frac{g}{L}}.
\end{displaymath}
This problem, which is quite simple to pose, has no analytic solution
in terms of elementary functions.  Instead, we will rely on numerical
techniques to solve it.\\

\section{Small Angle Approximation}

\begin{figure}[htbp]
\begin{center}
 \includegraphics[height=0.33\textheight]{figs/pendulum/sho.pdf} 
\caption{The small angle approximation for pendulum motion with period $T_L=1~\rm s$}.
\label{fig:sho}
\end{center}
\end{figure}


Analytic solutions are the precious gems that we use to validate our
numerical techniques.  We'll start our analysis in a region where we
can solve the problem analytically.  For small displacements of the
pendulum, $\theta$ is small, and so:
\begin{displaymath}
\sin \theta \approx \theta
\end{displaymath}
and
\begin{displaymath}
\alpha = -\omega_L^2 \, \sin \theta \approx -\omega_L^2 \, \theta
\end{displaymath}
The differential equation which describes the motion is:
\begin{displaymath}
\frac{d^2\theta}{dt^2} = -\omega_L^2 \, \theta
\end{displaymath}
We showed in lecture that if the initial angular velocity is zero ($\omega_0 = 0$) and the initial theta position is $\theta_0$, then the solution is:
\begin{equation}
\label{eqn:sho}
\theta(t) = \theta_0 \cos(\omega_L \, t)
\end{equation}
as plotted in Fig.~\ref{fig:sho}.  The pendulum rocks back and forth
following a sine wave with period:
\begin{displaymath}
T_L = \frac{2 \pi}{ \omega_L} = 2 \pi \sqrt{\frac{L}{g}}
\end{displaymath}


\plot Calculate $\omega_L$ for a pendulum with $L=1~\rm m$ and $g=9.8~\rm m/s^2$.
What are the units of $\omega_L$?\\

\plot  Suppose you want to construct a pendulum such that for small
oscillations the period $T_L = 1~\rm s$.  What should be the value of
$\omega_L$?  What length $L$ should you use, assuming that $g=9.8~\rm
m/s$?\\

\newpage

\plot Reproduce the plot in Fig.~\ref{fig:sho}.  Set the
constants $T_L$, $\omega_L$, $\theta_0$, and $N$ as follows:
\begin{python}
TL = 1          # period in seconds 
wL = 2*np.pi/TL # small-angle angular frequency of pendulum 
A  = 0.2        # theta at t=0 (amplitude)
\end{python}
You must complete this problem {\bf without using an explicit for
  loop}.  To set the $y$ axis label to the fancy $\theta$ do:
\begin{python}
plt.ylabel("$\\theta$")
\end{python}.

\section{The Failure of the Euler Method}

The Euler equations for angle $\theta$ and angular velocity $\omega$ are:
\begin{eqnarray}
  \theta_{n+1} &=& \theta_n + \tau \, \omega \\
  \omega_{n+1} &=& \omega_n + \tau \, \alpha 
\end{eqnarray}
where $\alpha$ is the angular acceleration and $\tau$ is the time step.\\

\plot Implement a function
\begin{python}
def euler(tau, theta, omega, alpha):
   # your code here
   return theta, omega
\end{python}
which implements one iteration of the Euler method.
Test your \pyth{euler} function with these test values:
\begin{python}
print(np.around(euler(-0.01, -0.28, -0.30, -0.06),3))
print(np.around(euler( 0.94,  0.32, -0.85, -0.86),3))
print(np.around(euler( 0.92, -0.16,  0.38, -0.32),3))
print(np.around(euler( 0.31,  0.12, -0.91, -0.76),3))
print(np.around(euler(-0.14,  0.96,  0.66, -0.73),3))
\end{python}
which should produce the following output:
\begin{verbatim}
[-0.277 -0.299]
[-0.479 -1.658]
[0.19  0.086]
[-0.162 -1.146]
[0.868 0.762]
\end{verbatim}
You'll use this now thoroughly tested function more below.  Don't change it!\\

\plot  Apply the Euler method to the problem of small oscillations of a pendulum with $T_L=1~\rm s$ as in Fig.~\ref{fig:sho}.  First, set the parameters of your code just as before:
\begin{python}
TL = 1          # period in seconds 
wL = 2*np.pi/TL # small-angle angular frequency of pendulum 
A  = 0.2        # theta at t=0 (amplitude)
\end{python}
Then implement the Euler method as follows:
\begin{algorithm}
  $\theta$ := A
  $\omega$ := 0
  $\tau$ := 0.0003
  Create an empty array tjth which will contain $\theta$ positions of the trajectory
  Append $\theta$ to the array tjth.
  Repeat $N$ times:
     Calculate $\alpha = -\omega_L^2 \theta$
     Update $\theta$ and $\omega$ for acceleration $\alpha$ and time step $\tau$ by calling euler().
     Append $\theta$ to the array tjth.
  Create an array t containting $N+1$ appropriately spaced time values.
  Plot tjth versus t
\end{algorithm} 
As always, first debug your code using a small value for $N$.  Then,
you should reproduce something that closely resembles Fig.~\ref{fig:sho} with
$N=10000$. \\

\plot Repeat the exercise above (you can use cut and paste) with $\tau=0.01$ and $N=1000$.  Yikes!  Is energy conserved?\\

\section{The Verlet Method}

The Verlet Equation for this problem is:
\begin{eqnarray}
  \theta_{n+1} &=& 2 \theta_n - \theta_{n-1} + \tau^2 \, \alpha
\end{eqnarray}
Notice that with the Verlet method, we will not need to calculate angular
velocity $\omega$ in order to get the $\theta$ trajectory.\\

\plot Implement a function
\begin{python}
def verlet(tau, theta, oldth, alpha):
   # your code here
   return theta
\end{python}
which returns $\theta_{n+1}$ from $\theta_n=$theta and
$\theta_{n-1}=$oldth using the verlet method.
Test your \pyth{verlet} function with these test values:
\begin{python}
print(np.around(verlet(-0.01, -0.28, -0.30, -0.06),3))
print(np.around(verlet( 0.94,  0.32, -0.85, -0.86),3))
print(np.around(verlet( 0.92, -0.16,  0.38, -0.32),3))
print(np.around(verlet( 0.31,  0.12, -0.91, -0.76),3))
print(np.around(verlet(-0.14,  0.96,  0.66, -0.73),3))
\end{python}
which should produce the following output:
\begin{verbatim}
-0.26
0.73
-0.971
1.077
1.246
\end{verbatim}
You'll use this now thoroughly tested function more below.  Don't change it!\\

\plot  In a previous exercise you showed that the Euler method, when applied to the problem of small oscillations of a pendulum with $T_L=1~\rm s$ for $\tau=0.01$ and $N=1000$, is unstable.  Instead, apply the Verlet method.  You can reuse (by copying and pasting) much of your code from that previous exercise with a few changes:
\begin{itemize}
 \item You will call your \pyth{verlet} function instead of \pyth{euler}.
 \item You no longer need to keep track of $\omega$ (\pyth{omega})
 \item You will now have to keep track of two $\theta$ values at all times.  At each updat:
\begin{displaymath}
   (\theta_n, \theta_{n-1}) \rightarrow  (\theta_{n+1}, \theta_{n})
\end{displaymath}
 \item You can start things off with \pyth{oldth = theta = A}
\end{itemize}
With this method, you should produce many oscillations with no sign of instability.\\

\plot So far, we have been using the small angle approximation.  Modify your code to use the exact formula for the angular acceleration $\alpha = \omega_L^2 \, \sin \theta$ and set the initial position to $A=2$.  This is a trivial change to your numerical simulation, but it  makes an analytic solution impossible!\\


\plot (Optional Challenge) Run you Verlet analysis for $N=1000$ steps
for $A=3$ and set $\tau$ appropriately so that you see a bit more than
one period of motion.  Plot the trajectory as a black line and read
off the period $T$.  Superimpose a plot of a cosine with amplitude $A$
and period $T$.\\

\plot (Optional Challenge) We've been starting things off with
approximately zero angular velocity by setting
\pyth{oldth = theta = A}.
You can add velocity to the initial state with:
\begin{python}
  V = 1
  theta = A
  oldth = A - tau*V
\end{python}
Give the pendulum enough of a whack that it reaches all the way to top
and keeps going.  When plotting this trajectory, you can use \pyth{np.mod} to
keep the theta values in the range from -$\pi$ to $\pi$ if you want.\\

\plot (Optional Challenge) Fix the Euler method for this pendulum
problem by forcing energy to be conserved at each step.  Compare your
results to the Verlet method.\\

















\chapter{Roots and Integrals}

\section{Introduction}

In this lab, we will implement numerical alogirthms for root-finding
and integration.  We will apply the root finding algorithm to a
classic problem from quantum mechanics: the bound-state energy levels
of particle in finite potential well.  We will apply numerical
integration to the calculation of the periods of a simple pendulum.

For a faster but more challenging path, you can complete only the
problems (but including the optional challenges) in
Sections~\ref{sec:pbox} and \ref{sec:romberg}.

\section{Roots of a Linear Function}

It is usually best to start simple when developing code, so we'll develop our algorithms using a simple linear function with a root at $x=3$.\\

\plot Implement $f(x)=5\,(x-3)$ as a python function
\begin{python}
def f(x):
   # your code ...
\end{python}
Calculate the derivative of $f$ and implement it as the function:
\begin{python}
def fp(x):
   # your code ...
\end{python}
Check you code with:
\begin{python}
for x in [-1,3,4]:
    print(f(x), fp(x))
\end{python}
which should return:
\begin{verbatim}
-20 5
0 5
5 5
\end{verbatim}.

\section{Bisection Method}

In the bisection method, we start from values $a_1$ and $b_1$ where
$f(a_1) \cdot f(b_1) \leq 0$.  This condition insures that the range
$[a_1, b_1]$ contains at least one root.  The bisection method halves
the interval with each iteration, chosing the half that contains at
least one root.  Given $a_n$ and $b_n$, we calculate:
\begin{displaymath}
c_n = (a_n + b_n)/2
\end{displaymath}
we then determine:
\begin{displaymath}
a_{n+1}=
\begin{cases}
a_n & f(a_n) \cdot f(c_n) \leq 0 \\
c_n & \rm (otherwise) \\
\end{cases}
\end{displaymath}
and
\begin{displaymath}
b_{n+1}=
\begin{cases}
c_n & f(a_n) \cdot f(c_n) \leq 0\\
b_n & \rm (otherwise) \\
\end{cases}
\end{displaymath}

\plot Implement the python function
\begin{python}
def bisection(f, a, b):
   # your code  
   return a, b # updated values
\end{python}
which, given the function f=$f(x)$ and interval defined by a=$a_n$
b=$b_n$, returns the smaller interval defined by a=$a_{n+1}$ and
b=$b_{n+1}$ using the bisection algorithm.  Test your code with:
\begin{python}
print(bisection(f,0,4))
print(bisection(f,3,4))
\end{python}
which should have output:
\begin{verbatim}
(2.0, 4)
(3, 3.5)
\end{verbatim}

\section{Newton's Method}

We can use Newton's method to find the roots of a function $f(x)$ if we know its derivative $f'(x)$.
From our current best estimate for the root $x_n$ we calculate a better estimate as:
\begin{displaymath}
x_{n+1} = x_n - \frac{f(x_n)}{f'(x_n)}.
\end{displaymath}

\plot Implement the python function:
\begin{python}
def newton(f,fp,x):
    # your code
    return x # updated value
\end{python}
which, given the function f=$f(x)$, it's derivative fp=$f'(x)$, and the current best estimate for the root x=$x_n$, 
returns the improved estimate $x_{n+1}$.  Test your code as:
\begin{python}
for x in [-100,5,1000]:
   print(newton(f,fp,x))
\end{python}
which should return the values $3$, $3$, and $3$.  Why does one single iteration of Netwon's method find the root in this case?

\section{Secant Method}

When we do not know the derivative a function, we can use the secant
method.  The secant method provides an improved estimate for the root
$x_{n+1}$ from the previous two estimates $x_n$ and $x_n-1$ which are used to 
estimate the derivative and then apply Newton's method:

\begin{displaymath}
x_{n+1} = x_n - f(x_n) \frac{x_n-x_{n-1}}{f(x_n) - f(x_{n-1}}
\end{displaymath}\\

\plot Implement the python function:
\begin{python}
def secant(f,a,b):
   # your code here  
   return b,c  
\end{python}
which, given the function f=$f(x)$, and two previous estimates for the root a=$x_{n-1}$ and b=$x_n$, returns the updated estimates $b=x_n$ and $c=x_{n+1}$ using the secant method.  Test your code as:
\begin{python}
print(secant(f,0,4))
print(secant(f,5,3))
\end{python}
which should return
\begin{verbatim}
(4, 3.0)
(3, 3.0)
\end{verbatim}

\section{Roots of a Quadratic Function}

In this section, we will test your root finding algorithms on a quadratic equation:
\begin{equation}
  g(x) = (x-1) \, (x-4)
\end{equation}

\plot Define a python function \pyth{g(x)} which returns the value
$g(x)$.  Calculate the derivative $g'(x)$ analytically, and define
\pyth{gp(x)} which returns $g'(x)$.  Plot $g(x)$ and $g'(x)$ for
$0<x<5$.  Be sure to include axis labels and a legend. \\

\plot Apply five iterations of your function \pyth{bisection} to the range (0,2.5).  Do you approach the correct root?   \\

\plot  Starting from estimates a=0 and b=2.5, apply five iterations of your function \pyth{secant}.  Do you approach the correct root? \\


\plot Starting from the value x=2.5, apply five iterations of your function \pyth{newton}.  Do you approach a root? \\

\section{Particle in a Finite Potential Well}
\label{sec:pbox}

Suppose a particle of mass $m$ and energy $E$ is located in a finite potential well of form:
\begin{displaymath}
  V(x) =
  \begin{cases}
    V_0 & x \leq -L/2 \\
    0 & -L/2 < x < L/2 \\
    V_0 & L/2 \leq x \\
   \end{cases}
\end{displaymath}
We are going to consider the interesting situation where $E<V_0$, that
is, when the particle is bound by the potential.  For simplicity, we
will also assume the particle is in a symmetric state, such that wave
function $\psi$ has the property $\psi(-x)=\psi(x)$.  In this case, the
solutions to the Schrodinger Equation satisfy the transcendental
equation:
\begin{equation}
\label{eqn:transcendental}
v \tan v = \sqrt{v_0^2 - v^2}
\end{equation}
where 
\begin{displaymath}
v = L \sqrt{\frac{m \, E }{2 \hbar}}
\end{displaymath}
and
\begin{displaymath}
v_0^2 = \frac{m \, L^2 \, V_0}{2 \hbar} 
\end{displaymath}
are both dimensionless quantities, related to the particle energy $E$
and the potential energy of the well $V_0$.  If you haven't
yet encountered this essential problem from quantum mechanics, you can
just take my word about Equation~\ref{eqn:transcendental} for now.

\begin{figure}[htbp]
\begin{center}
\includegraphics[width=0.65\textwidth]{figs/roots/transcendental.pdf} 
\caption{Graphical solution to Equation~\ref{eqn:transcendental} for $v_0=10$}
\label{fig:transcendental}
\end{center}
\end{figure}

Most quantum mechanical textbooks suggest that the solutions to
Equation~\ref{eqn:transcendental} may be obtained ``graphically'', by
plotting $u = v \tan v$ and $u = \sqrt{v_0^2 - v^2}$ and observing where
they intersect.  This approach is illustrated in
Fig.~\ref{fig:transcendental} for $v_0=10$.  You can see from the plot
that there are four places where these curves intersect.  In a
classical system, the particle could have any $E<V_0$, but in quantum
mechanical system, the particle may have only one of the four energy
values corresponding to the four intersection points. The number of
interesections, and where they occur, depends on the particular value
of $v_0$.\\

\plot Reproduce Fig.~\ref{fig:transcendental}. If you attempt to plot
the dashed red lines all at once, you will introduce
superflous vertical lines from when $\tan v$ switches from $+\infty$ to
$-\infty$.  To avoid this, plot each continuous region of $u = v \tan
v$ separately, in the regions $(-\pi/2,\pi/2)$, then $(\pi/2,3\pi/2)$,
and so on.  You may find it useful to remove problematic end points
from a numpy array by slicing them off like this \pyth{x=x[1:-1]}.
Instead of the math formulas, your legend may use the simpler labels LHS and RHS
, which refer to the left hand side and right hand side of
Equation~\ref{eqn:transcendental}.\\

Determining the bound state energy levels of quantum system is of
fundamental importance.  For just one example, we are able to
determine the chemical composition of stars using spectroscopy.  The
light frequencies which are readily absorbed by each element are
determined from the difference in the bound state energy levels.  In
this section, we will use numerical techniques to accurately determine
the values of $v$ which satisfy Equation~\ref{eqn:transcendental}.
This is equivalent to finding the bound-state energy levels.  We will
do so by finding the roots of the equation:
\begin{equation}
\label{eqn:hx}
h(x) = v \tan v - \sqrt{v_0^2 - v^2}
\end{equation}\\

\plot Define a function:
\begin{python}
def h(x):
   # your code   
\end{python}
which implements $h(x)$ from Equation~\ref{eqn:hx} for $v_0=10$.  Test it with:
\begin{python}
for v in [1,1.5,2,4.5]:
    print(np.around(h(v),2))
\end{python}
which should output the values -8.39, 11.27, -14.17, and 11.94.\\

\plot For $v_0=10$, determine the four roots of $h(x)$ using the bisection algorithm.\\

\plot For $v_0=10$, determine the four roots of $h(x)$ using the
secant algorithm.  To get reliable performance, you might want to
start by applying five iterations of the bisection algorithm.\\

\plot Calculate the derivative of $h(x)$ analytically, and implement it for $v_0=10$ as the python function \pyth{hp(x)}. Test you function as:
\begin{python}
for v in [1,1.5,2,4.5]:
    print(np.around(hp(v),2))
\end{python}
which should output the values 5.08, 314.03, 9.57, and 106.41.

\plot For $v_0=10$, determine the four roots of $h(x)$ using Newton's method. \\

\plot (Optional Challenge) The antisymmetric solutions to the Schrodinger Equation satisfy the equation:
\begin{displaymath}
- v \cot v = \sqrt{v_0^2 - v^2}
\end{displaymath}
Find the solutions for $v$ using Newton's method.\\

\plot (Optional Challenge) The Wikipedia article on the finite potential well lists the solutions to $v$ for $v_0^2=20$.  Verify their results.

\section{Trapezoid Method}

The trapezoid method approximates the definite integral
\begin{displaymath}
I = \int_a^b f(x) \, dx
\end{displaymath}
from the function evaluated at $n$ evenly spaced points
\begin{displaymath}
I_T =  \frac{h}{2}(f(a) + f(b)) + h \, \sum_{i=1}^{n-1} f(x_i)
\end{displaymath}
where
\begin{displaymath}
h = \frac{b-a}{n-1}
\end{displaymath}
and
\begin{displaymath}
x_i = a + h \, i 
\end{displaymath}
The truncation error is:
\begin{displaymath}
I-I_T = \mathcal{O}(h^2).
\end{displaymath}
Notice that the sum is over the interior points, which have twice the
weight of the end points at $a$ and $b$.

\plot Implement the python function:
\begin{python}
def trapezoid(f,a,b,n):
   # your code
   return sum
\end{python}
Which returns the integral of f=$f(x)$ from a to b, using the
trapezoid method from $n$ evenly spaced points.
Test your code with:
\begin{python}
print(np.around(trapz(np.sin,0,np.pi/2,2),2))
print(np.around(trapz(np.sin,0,np.pi/2,3),2))
print(np.around(trapz(np.sin,0,np.pi/2,4),2))
\end{python}
which should output the values 2.36, 1.73, and 1.5.

\section{Iterative Trapezoid Method}

In the iterative trapezoid method, we halve the spaces between points
during each interation, so that:
\begin{displaymath}
h_1 = (b-a), \; h_2 = \frac{b-a}{2}, \; \ldots \; , \; h_m = \frac{b-a}{2^{m-1}}
\end{displaymath}
Because the interval size $h_{m+1}$ is one half the previous interval
$h_m$, we can avoid evaluating previously evaluated $x$ values by
resusing the previous iteration:
\begin{displaymath}
I_{m+1} = \frac{1}{2} \, I_{m} + h_{m+1}\sum_{i=1}^{2^{(m-1)}} f(a+(2i-1)h_{m+1}).
\end{displaymath}
Take care when reading that the upper limit on the sum is $2^{(m-1)}$
not $2\,(m-1)$.  A major benefit of this approach is that the
difference between $I_{m+1}$ and $I_m$ can be used to estimate the
truncation error.\\

\plot Implement the python function:
\begin{python}
def itertrap(f,a,b,s,m):
   # your code
   return s, m # updated values! 
\end{python}
Which calculates one iteration of the iterative trapezoid method for
the integral of f=$f(x)$ from a to b, starting from s=$I_{m}$. Returns
the updated values $\rm{s} \to I_{m+1}$ and $\rm{m} \to m+1$.
Test your code with:
\begin{python}
print(np.around(itertrap(np.sin,np.pi/2,np.pi,0.00,1),2))
print(np.around(itertrap(np.sin,np.pi/2,np.pi,0.56,2),2))
print(np.around(itertrap(np.sin,np.pi/2,np.pi,0.79,3),2))
\end{python}
which should output:
\begin{verbatim}
[0.56 2.  ]
[0.79 3.  ]
[0.9 4. ]
\end{verbatim}

\plot Calculate the definite integral:
\begin{displaymath}
I = \int_0^\pi \sin(x) \, dx
\end{displaymath}
using the iterative trapezoid method.  Notice that $I_1 = 0$ so start things off with $s=0$ and $m=1$.  Use the update idiom:
\begin{python}
s,m = itertrap(f, a, b, s, m)
\end{python}
and iterate until the estimated truncation error is less than $\sigma=10^{-6}$.


\section{Period of a Pendulum}

In this section we will consider the oscillation of a pendulum of length $L$ in a graviational field with magnitude $g$.  For small angles, the period of a pendulum oscillation is approximately a constant value:
\begin{displaymath}
T_0 = 2\pi \frac{L}{g}
\end{displaymath}
For oscillations at larger angles, the period of the oscillation cannot be calculated analytically. However, it can be expressed as a definite integral:
\begin{equation}
\label{eqn:elliptic}
\frac{T}{T_0} = \frac{2}{\pi}\int_0^{\pi/2}\frac{1}{\sqrt{1-k^2\sin u}}\,du 
\end{equation}
where:
\begin{displaymath}
k = \sin \frac{\theta_0}{2}
\end{displaymath}

\plot Estimate the ratio $T/T_0$ for $\theta_0=1$ using the iterative
trapezoid rule.  Iterate until the estimated truncation error is less
than $10^{-6}$.  Note that in this case $I_1$ is not zero, so you will
have to calculate the starting value for $s$.

\section{Romberg Integration}
\label{sec:romberg}

The section contains a challenging problem which will only represent a
small part of your grade (unless you have opted to take the fast and
challenging path.)  You should attempt to tackle if for yourself.
Your instructors will not help you solve this problem!  You should be
able to tackle this problem if you break it into steps and test each
step carefully, using all of the techniques you have learned this
quarter.  Don't be discourage if you cannot comnplete it successfully,
this is meant to be a challenge!

The Romberg Method is an ingenious approach to integration which
combines the iterative trapezoid rule with Richardson appoximation.
This is an example of a high-end algorithm which converges extremely
rapidly.  The method iteratively constructs the matrix $R$
\begin{displaymath}
\begin{pmatrix}
R_{1,1} &         &         &         & \\
R_{2,1} & R_{2,2} &         &         & \\
R_{3,1} & R_{3,2} & R_{3,3} &         & \\
\vdots  & \vdots  & \vdots  & \ddots  & \\
R_{m,1} & R_{m,2} & R_{m,3} & \cdots  & R_{m,m} \\
\end{pmatrix}
\end{displaymath}
The values in the leftmost column are constructed using the iterative trapezoid method:
\begin{displaymath}
R_{m,1} = I_m  
\end{displaymath}
and the entries to the right are calculated iteratively using Richardson's approximation:
\begin{displaymath}
R_{m,j+1} = \frac{4^j R_{m,j} - R_{m-1,j}}{4^j-1} 
\end{displaymath}
The best estimates are the values along the diagonal.

Note that each row is calculated using only values from previous row, so we will tackle this problem row by row.
Given:
\begin{displaymath}
\begin{pmatrix}
R_{1,1} \\
\end{pmatrix}
\end{displaymath}
the next iteration should return:
\begin{displaymath}
\begin{pmatrix}
R_{2,1} & R_{2,2} \\
\end{pmatrix}
\end{displaymath}
and given:
\begin{displaymath}
\begin{pmatrix}
R_{2,1} & R_{2,2} \\
\end{pmatrix}
\end{displaymath}
the next iterations should return:
\begin{displaymath}
\begin{pmatrix}
R_{3,1} & R_{3,2} & R_{3,3}\\
\end{pmatrix}
\end{displaymath}

\newpage

\plot (Challenging, not optional) Define  python function
\begin{python}
def romberg(f,a,b,R):
   # your code
   return R # updated value
\end{python}
which given the $m$th row as array R:
\begin{displaymath}
R = 
\begin{pmatrix}
R_{m,1} & R_{m,2} & \cdots & R_{m,m}\\
\end{pmatrix}
\end{displaymath}
returns the updated array:
\begin{displaymath}
R = 
\begin{pmatrix}
R_{m+1,1} & R_{m+1,2} & \cdots & R_{m+1,m+1}\\
\end{pmatrix}
\end{displaymath}
Test your code by calculating the definite integral:
\begin{displaymath}
I = \int_0^\pi \sin(x) \, dx
\end{displaymath}
by constructing four rows of the Romberg matrix.\\

\plot (Challenging, optional) Use Romberg integration to calculate the
period of a pendulum with length $L$ in a gravitational field of
strength $g$ which comes to rest at $\theta = \theta_0$, by
numerically integrating Equation~\ref{eqn:elliptic} using the Romberg
method.  Stop for a truncation error less than $10^{-6}$.


\chapter{Ideal Gas}

\section{Introduction}

In this lab, you will construct
a Monte Carlo simulation of a 2-D ideal gas, and show that the
molecular velocities follow the Maxwell-Boltzman distribution.

%For a faster but more challenging path, complete only the problems
%(including optional challenge problems) in Sections~\ref{sec:mcint},
%\ref{sec:mbdist}, and \ref{sec:idealgas}.


\section{Ideal Gas in Two Dimensions}

\begin{figure}[htbp]
\begin{center}
\includegraphics[width=0.65\textwidth]{figs/ideal_gas/mbspeed.pdf} \\
\caption{The Maxwell-Boltzmann distribution, for three different
  temperatures, using the system of units chosen for the numerical
  simulation.}
\label{fig:mbspeed}
\end{center}
\end{figure}

\noindent
For the remainder of this lab, we will calculate the statistical
properties of a simulated ideal gas and compare to the theoretical
prediction.  Our ideal gas will be composed of molecules with mass $m$
and held at temperature $T$.  To keep things simple, we will consider
a two dimensional gas.

A fundamental result from statistical mechanics is that the
propability of finding a single gas molecule in a state with energy
$E$ is proportional to the Boltzman factor:
\begin{displaymath}
P(E) \; \propto \; \exp\left(-\frac{E}{k_{\rm B}T}\right)
\end{displaymath}
where $k_{\rm B}$ is Boltzmann's constant.  Since the ideal gas has no
interactions, its energy is purely kinetic energy, and in two
dimensions this is given by:
\begin{displaymath}
E = \frac{1}{2} \, m \, (v_x^2 + v_y^2)
\end{displaymath}
So we can conclude:
\begin{displaymath}
P(v_x, v_y) \; \propto \; \exp\left(-\frac{m\,(v_x^2+v_y^2)}{2k_{\rm B}T}\right)
=
\exp\left(-\frac{m\,v_x^2}{2kT}\right) \cdot \exp\left(-\frac{m\,v_y^2}{2kT}\right)
\end{displaymath}
We can infer that the probability density for the component of velocity in $x$ direction is given by:
\begin{equation}
  \label{eqn:mbvx}
P(v_x) = \sqrt{\frac{m}{2 \pi k_{\rm B} T}} \exp\left(-\frac{m v_x^2}{2k_{\rm B} T}\right)
\end{equation}
where we have calculated the normalization constant such that:
\begin{equation*}
\int_{-infty}^{+\infty} P(v_x) dv_x = 1.
\end{equation*}
Similary:
\begin{equation}
  \label{eqn:mbvy}
P(v_y) = \sqrt{\frac{m}{2 \pi k_{\rm B} T}} \exp\left(-\frac{m v_y^2}{2k_{\rm B} T}\right)
\end{equation}

Now we shall find the PDF associated with a particular speed $v$.  We consider the infinitesimal probability for a particular velocity
\begin{eqnarray*}
P(v_x) P(v_y) \, dv_x \, dv_y
  &=& \frac{m}{2 \pi k_{\rm B} T} \exp\left(-\frac{m (v_x^2+v_y^2)}{2k_{\rm B} T}\right) \, dv_x \, dv_y \\
  &=& \frac{m v}{2 \pi k_{\rm B} T} \exp\left(-\frac{m v^2}{2k_{\rm B} T}\right) \, d\theta \, dv \\
\end{eqnarray*}
where we have changed to polar coordinates $v$ and $\theta$ in the usual manner with area differential $dv_x \, dv_y = v \, dv \, d\theta$.  This allows us to read off the probability density in polar coordintes:
\begin{equation*}
P(v, \theta) = \frac{m v}{2 \pi k_{\rm B} T} \exp\left(-\frac{m v^2}{2k_{\rm B} T}\right) 
\end{equation*}
Integrating over all possible directions $\theta$, we obtain:
\begin{eqnarray}
P(v) &=& \int_0^{2\pi} P(v,\theta) d\theta \nonumber \\
     &=& \int_0^{2\pi} \frac{m v}{2 \pi k_{\rm B} T} \exp\left(-\frac{m v^2}{2k_{\rm B} T}\right) \nonumber \\
P(v) &=& \frac{m v}{k_{\rm B} T} \exp \left(-\frac{m v^2}{2k_{\rm B} T}\right) \label{eqn:mbv}\\
\nonumber
\end{eqnarray}
which is the Maxwell-Boltzmann distribution for an ideal gas in two
dimensions.  This is the probability density for a gas molecule to
have speed $v$. It is illustrated in Fig.~\ref{fig:mbspeed}.  In this
lab, we will create a simple numerical simulation of an ideal gas and
verify that the velocity of the gas follows this distribution.

\section{System of Units}

Choosing an effective system of units is essential for building a
well-behaved numerical simulation.  By now you have hopefully learned
the wisdom of solving problems analytically using only variables,
plugging actual numbers into your equations only if necessary and only
at the very end.  Numerical techniques generally depend on using
actual numerical values, but by making a wise choice for a
computational system of units, we can recover the same universality
and clarity that variables provide to analytic solutions.

Consider the Maxwell-Boltzmann distribution, which involves the following SI values:
\begin{itemize}
\item Boltzmann's constant: $k_{\rm B} = 1.38 \times 10^{-23}~\rm J/K$
\item Molecular masses: e.g. $N_2$ with $m =  4.65 \times 10^{-26}~ \rm kg$.
\item Temperature: e.g. room temperature $T = 293~\rm K$.
\end{itemize}
The smallest number greater than zero that a computer can represent
with a single-precision floating point number is approximately
$10^{-38}$. Representing the SI value of Boltzmann's constant at
$10^{-23}$ uses a large fraction of this precision before we even begin
our calculation.  Numerical algorithms using floating point numbers
work best when the values involved in the calculation are near one.

It is usually best, therefore, to devise an alternate system of units
for any numerical simulation which keeps the values of variables of
interest as near one as possible.  We will call this the numerical
system of units.

To start, we choose a reference temperature near the temperature
we would like to simluate, say $T_0 = 293~\rm K$.  All temperatures in
the simulation will be in units of this reference temperature.  So a
temperature \pyth{T=1.2} in the program will be $1.2~T_0 = 352~\rm K$
in SI units.  Our model also includes mass, so we choose a reference
mass near the mass of the molecules we will be simulating, say $M_0 =
4.65 \times 10^{-26}~ \rm kg$.  A mass {\tt m=2.1} in our program would have
an SI value value of $2.1~M_0 = 9.8 \times 10^{-26}~ \rm kg$.

The physics we will simulate involves Boltzmann's constant $k_{\rm B}$
which will have a value of one in our program.  This sets the
reference energy from our reference temeperature.  For example, an
energy {kT == 3} in our program will have an SI value of $k_{\rm B}
T = 3~k_B~T_0 = 1.21 \times 10^{-20}~J$.  The reference energy and
reference mass together define a reference velocity:
\begin{displaymath}
V_0 = \sqrt{\frac{k_b T_0}{M_0}} = 295~ \rm m/s.  
\end{displaymath}  

The only time the actual values choosen for the numerical system of
units are needed is if you need to convert inputs in SI units to the
numerical system of units, or convert the results of your simulation
to SI units.  In this lab, we will specify all inputs and report all
results using the numerical system of units.  {\bf So there is no need
  for specific values such as $M_0 = 2.32 \times 10^{-25}~\rm kg$ to
  appear anywhere in your program.}  If such values do appear, outside
of comments, you are certainly making a mistake!  If you are living
entirely within the numerical system of units, there is no need to
even define $M_0$: this is how you recover universality in numerical
simulations.\\

\plot Using the numerical system of units:
\begin{python}
# computational system of units:
M  = 1 # mass of gas particles, M0 = 4.65E-26 kg
T  = 1 # Temperature of gas, T0 = 293 K
kb = 1 # Boltzmans constant
\end{python}
and the Maxwell-Boltzman distribution:
\begin{python}
  def mbspeed(v):
      return (M*v / (kb*T))*np.exp(-M*v**2/(2*kb*T))
\end{python}
plot the Maxwell-Boltzman distribution for v = 0 to $5~V_0$.  Compare with Fig.~\ref{fig:mbspeed}.\\

Notice how the relevant velocities for T=1 are near v=1.  This is sign
of good numerical system of units.  Notice also that Boltzmann's
constant or any other small or large numbers in SI units do not appear
anywhere in the code (only in comments).\\

\plot Suppose we set $M_0=10^{-15}~\rm kg$ instead.  Would your plot
in the previous exercise need to be changed?  What about
Fig.~\ref{fig:mbspeed}?  From Fig.~\ref{fig:mbspeed}, read off the
peak of the distribution for T = 2 T0.  What is that in SI units,
assuming $M_0=4.65 \times 10^{-26}~\rm kg$?  What if $M_0=10^{-15}~\rm
kg$?  Do we even need to specify $M_0$ if we do not care about the
specific SI values?  Is choosing a computational system of units with
variables near one the numerical analysis equivalent to solving a
problem using variables only?

\section{Collision Model}

At the heart of your numerical simulation is the collision model.  It
is the collisions of molecules that will allow your simulated gas to
reach thermal equilibrium.  We will use the simple elastic collision
of identical mass particles, as illustrated in Fig.~\ref{fig:collcms}, as our collision model.  We consider particles a and b with velocities $\vec{v_a}$ and
$\vec{v_b}$ in the lab frame.  The velocity of particle a in the CMS frame before the collision is
\begin{displaymath}
\vec{u} = \frac{\vec{v_a} - \vec{v_b}}{2}.
\end{displaymath}
and the velocity of particle b is $-\vec{u}$.

The collision rotates the velocity of particle a by the scattering angle $\theta$ so that the velocity $\vec{w}$ after the collision is
\begin{displaymath}
\begin{pmatrix}
w_x \\
w_y \\
\end{pmatrix}
  =
\begin{pmatrix}
\cos \theta  & -\sin \theta \\
\sin \theta  & \cos \theta \\
\end{pmatrix}
\,
\begin{pmatrix}
u_x \\
u_y \\
\end{pmatrix}
\end{displaymath}
The velocity of particle b after scattering is $-\vec{w}$.

In the lab frame, the velocity of molecule a changes by an amount:
\begin{displaymath}
\Delta \vec{v_a} = \vec{w} - \vec{u}  
\end{displaymath}
and the velocity of molecule b changes by an amount:
\begin{displaymath}
\Delta \vec{v_b} = (-\vec{w}) - (-\vec{u}) = -\Delta \vec{v_a}
\end{displaymath}

\begin{figure}[htbp]
\begin{center}
\begin{tikzpicture}
\draw[->, line width=1.5, blue] (-3,0) -- (-0.1,0);
\draw[->, line width=1.5, blue] (3,0)  -- (0.1,0);
\draw[->, line width=1.5, red] (0,0) -- (3*0.50,3*0.86) coordinate(A);
\draw[->, line width=1.5, red] (0,0) -- (-3*0.50,-3*0.86) coordinate(B);
\node[right] at (0.5,0.5) {$\theta$};
\node[left] at (-3,0) {a};
\node[right] at (3,0) {b};
\node[above] at (A) {a};
\node[below] at (B) {b};
\node[above] at (-1.5,0) {$\vec{u}$};
\node[above] at (1.5,0) {$-\vec{u}$};
\node[left] at (0.8,1.5) {$\vec{w}$};
\node[left] at (-0.8,-1.4) {$-\vec{w}$};
\end{tikzpicture}
\caption{The collision model in the center-of-mass:  incoming molecule $a$ with velocity $\vec{u}$ collides with the incoming particle $b$ of identical mass with velocity $-\vec{u}$.  Particle $a$ is scattered by angle $\theta$ and leaves with velocity $\vec{w}$, while particle $b$ leaves with velocity $\vec{w}$.  The magnitude of the final and initial velocities are the same:  $|\vec{u}| = |\vec{w}|$.}
\label{fig:collcms}
\end{center}
\end{figure}

\section{Implementing the Collision Model}

Our Python implementation for the collision will be computed in terms
of the components of the velocity vectors of molecule a and molecule
b:
\begin{eqnarray*}
\vec{v_a} &=& 
\begin{pmatrix}
a_x \\
a_y \\
\end{pmatrix} \\
\vec{v_b} &=& 
\begin{pmatrix}
b_x \\
b_y \\
\end{pmatrix} \\
\end{eqnarray*}
We'll use the Python variable names {\tt ax}, {\tt ay}, {\tt bx}, and {\tt by} to refer to $a_x$,  $a_y$,  $b_x$, and $b_y$.  

First calculate the $x$ and $y$ component of $\vec{u}$ as:
\begin{eqnarray*}
u_x &\equiv& \frac{a_x - b_x}{2} \\
u_y &\equiv& \frac{a_y - b_y}{2} \\
\end{eqnarray*}
Then compute the $x$ and $y$ component to the change in velocity of particle a and particle b:
\begin{eqnarray*}
  \Delta a_x &=& (\cos\theta - 1) \, u_x - \sin\theta \, u_y \\
  \Delta a_y &=& (\cos\theta - 1) \, u_y + \sin\theta \,  u_x \\
\end{eqnarray*}
Finally, update the $x$ and $y$ components of the particle velocities to their value after the collision:
\begin{eqnarray*}
  a_x &\to& a_x + \Delta a_x \\
  a_y &\to& a_y + \Delta a_y \\
  b_x &\to& b_x - \Delta a_x \\
  b_y &\to& b_y - \Delta a_y \\
\end{eqnarray*}

\newpage

\plot Implement the collision model as a Python function:
\begin{python}
def collide(ax,ay,bx,by,theta):
    # your code here
    return ax, ay, bx, by # updated values
\end{python}
which takes the lab frame veclocitys of particle a (ax,ay) and b (bx, by) and returns the lab frame velocities after scattering by an angle theta.  Test your code with some simple cases first, rotations by zero,quarter,half,and three quarters of a full turn:
\begin{python}
tau = 2*np.pi # Using 2 pi is like saying twice half-way...
# lab frame is cms, incoming on x axis:
print(np.around(collide(1,0,-1,0,0),2)+0)
print(np.around(collide(1,0,-1,0,tau/4),2)+0)
print(np.around(collide(1,0,-1,0,tau/2),2)+0)
print(np.around(collide(1,0,-1,0,3*tau/4),2)+0)
\end{python}
which should have the output:
\begin{verbatim}
[ 1.  0. -1.  0.]
[ 0.  1.  0. -1.]
[-1.  0.  1.  0.]
[ 0. -1.  0.  1.]
\end{verbatim}
And:
\begin{python}
# lab frame is cms, incoming on	y axis:
print(np.around(collide(0,1,0,-1,0),2)+0)
print(np.around(collide(0,1,0,-1,tau/4),2)+0)
print(np.around(collide(0,1,0,-1,tau/2),2)+0)
print(np.around(collide(0,1,0,-1,3*tau/4),2)+0)
\end{python}
which should have the output:
\begin{verbatim}
[ 0.  1.  0. -1.]
[-1.  0.  1.  0.]
[ 0. -1.  0.  1.]
[ 1.  0. -1.  0.]
\end{verbatim}

\newpage

When debugging code, start with simple test cases.  If you get the
wrong answer, it will be easy to see where the calculation is going
wrong.  Once your code works on the simple tests cases, start adding
more complexity:\\

\plot Test your collision function where the lab frame is not the CMS frame:
\begin{python}
# boost along x axis, incoming on y axis:
print(np.around(collide(1,1,1,-1,0),2)+0)
print(np.around(collide(1,1,1,-1,tau/4),2)+0)
print(np.around(collide(1,1,1,-1,tau/2),2)+0)
print(np.around(collide(1,1,1,-1,3*tau/4),2)+0)
\end{python}
which should output:
\begin{verbatim}
[ 1.  1.  1. -1.]
[0. 0. 2. 0.]
[ 1. -1.  1.  1.]
[2. 0. 0. 0.]
\end{verbatim} \vskip 0.25cm

\plot Test your collision function with randomly choosen values:
\begin{python}
# test with random values:
print(np.around(collide(6.24, 1.78, 3.35, 5.98, 3.19),2))
print(np.around(collide(4.07, 4.69, 1.61, 4.54, 2.46),2))
print(np.around(collide(5.28, 2.99, 4.77, 5.22, 3.15),2))
print(np.around(collide(2.84, 5.37, 5.47, 6.16, 1.59),2))
\end{python}
which should output:
\begin{verbatim}
[3.25 5.91 6.34 1.85]
[1.84 5.33 3.84 3.9 ]
[4.76 5.22 5.29 2.99]
[4.58 4.46 3.73 7.07]
\end{verbatim}

\section{Initializing the Simulated Ideal Gas}

We will be modeling an ideal gas by direct Monte Carlo simulation of
\pyth{NGAS} representative molecules.  The state of your simulation
will be completely contained in two numpy arrays {\tt vx} and {\tt
  vy}, each of length {\tt NGAS}, which contain the velocities of the
particles in units of $V_0 = \sqrt{k_b T_0 / M_0}$.  Remember, the
simulation uses a system of units that should keep velocities near 1,
so values such as 2.2, -3.1, 0.8, -0.01 are all likely, and correspond
to speeds of up to several hundred meters per second in SI units.  On
the other hand, the presence of extremely small values, like 5.3E-23,
and extremely large values like 1.2E18 and -8.2E28 are symptoms of
bugs.\\

\plot Start with a small number of molecules \pyth{NGAS=5}.  Use the
\pyth{np.random.uniform} function to initialize the \pyth{vx} and
\pyth{vy} arrays with random values choosen uniformly in the range
(-2,2).  Print your arrays and see if the output is reasonable.\\

Our simulation is going to grow to contain at least 1000 molecules, so
printing the values of each molecule is not going to work.  So we will want to visualize it as a histogram.\\

\plot Increase NGAS to 1000, and plot the $v_x$ array as a histogram:
\begin{python}
hvx,bins = histogram(vx,bins=20,range=(-5,5))
cbins = (bins[1:]+bins[:-1])/2
plt.errorbar(cbins,hvx,yerr=np.sqrt(hvx),fmt="o")
\end{python}
~\\ \vskip -0.5cm

\plot Plot the $v_y$ distribution as a histogram.\\

Your histograms should reveal that your distribution of velocities is
flat, just like in Fig.~\ref{fig:histpdf}.  The gas molecules have not
yet reached thermal equilibrium with each other!

\section{Collisions of an Ideal Gas}

To reach thermal equilibrium, you'll need to simulate collisions
betweens pairs of molecules in your gas.  For each collision, do the following:
\begin{itemize}
 \item Choose two molecules at random as particles $a$ and $b$. (See {\tt np.random.choice}.)
 \item Choose a random value $\theta$ uniformly in the range $[0,2\pi]$ (See {\tt np.random.uniform}.)
 \item Call your collision function with components of the velocity vectors for particles $a$ and $b$ and the scattering angle $\theta$.
 \item Update the velocity of particles $a$ and $b$ from the return value of your collision funciton
\end{itemize}   
For this model, you will need about 10 times as many collisions as gas
molecules in order to reach thermal equilibrium.\\

\plot For \pyth{NGAS=1000} moleculres, simulated \pyth{NCOLL=10000}
collisions using the algorithm described above.  Plot the vx and vy
distributions as a histogram:
\begin{python}
hvx,bins = histogram(vx,bins=20,range=(-5,5))
hvy,bins = histogram(vy,bins=20,range=(-5,5))
cbins = (bins[1:]+bins[:-1])/2
plt.errorbar(cbins,hvx,yerr=np.sqrt(hvx),fmt="bo",label="vx")
plt.errorbar(cbins,hvy,yerr=np.sqrt(hvx),fmt="ro",label="vy")
plt.xlabel("Velocity (V0)")
plt.ylabel("Molecules")
plt.legend()
\end{python}

\section{Temperature of an Ideal Gas}

The temperature of the gas is related to the mean kinetic energy by:
\begin{equation}
\label{eqn:kt}
k_b T = m \, \frac{\braket{v_x^2} + \braket{v_y^2}}{2}  
\end{equation}  
You can estimate $\braket{v_x^2}$ from your simulation as {\tt np.mean(vx**2)}.\\

\plot Estimate $kT$ of the gas using Equation~\ref{eqn:kt} before and after simulating collisions.
The values should remain near the expected value 4/3.


\section{The Maxwell-Boltzmann Distribution}
\label{sec:mbdist}


In this section, you'll reproduce the instructor plots of
Fig.~\ref{fig:mbinst} using your own numerical simulation.\\

\begin{figure}[htbp]
\begin{center}
  \begin{tabular}{cc}   
  \includegraphics[width=0.45\textwidth]{figs/maxwellboltzman/mbinstv.pdf} &
  \includegraphics[width=0.45\textwidth]{figs/maxwellboltzman/mbinstvxvy.pdf} \\
  (a) & (b) \\
  \end{tabular}
\caption{Instructor plots.}
\label{fig:mbinst}
\end{center}
\end{figure}

\plot After your simulation reaches equilibrium, plot two histograms,
one with $v_x$ and one with $v_y$, with an approriate range and 10
bins.  Directly compare your histograms with the PDF from
Equation~\ref{eqn:mbvx}, scaled appropriately.  The results should
resemble the right side of Fig.~\ref{fig:mbinst}, which were produced
with {\tt NGAS=10000}.\\

\plot After your simulation reaches equilibrium, fill a histograms
with the magnitude of the velocity $v$, with an approriate range and
10 bins.  Compare with the prediction from Equation~\ref{eqn:mbv}.
The results should resemble the left side of Fig.~\ref{fig:mbinst},
which were produced with {\tt NGAS=10000}.\\

\section{Ideal Gas Law}
\label{sec:idealgas}

%This section contains only optional challenge problems.  Only minimal
%instructor help will be available for these problems.\\

\plot (Optional Challenge)  Implement a function which scales the
velocities $v_x$ and $v_y$ by an appropriately choosen factor to set
the temperature of the gas to any desired temperature $T$.  Set the
temperature to $1~T_0$, and show that the gas follows the
Maxwell-Boltzman distribution for $T=1~T_0$.\\

The ideal gas law for a 2-D gas of $N$ molecules at temperature $T$ is
\begin{displaymath}
  Q A = N \, k_{\rm b} T
\end{displaymath}
where $Q$ is the 2-D analog for pressure: a force per unit distance.
We'll consider a rectangular region of gas with height $H$ and length
$L$.  In this case, the ideal gas law becomes:
\begin{displaymath}
  F L = N \, k_{\rm b} T
\end{displaymath}
where $F$ is the total outward force acting on a each boundary of height $H$.

Our computational system of units will be extended to include a
distance $L_0$.  As long as we don't need to convert to SI units,
there is no need for us to specify $L_0$ in meters.  The units of time
are $L_0/V_0$.  The time it takes for a molecule to travel a distance
$L$ is approximately $1~L_0/V_0$, so we will step our simulation in
steps of about $\tau=0.01~L_0/V_0$.

To model the ideal gas law, we will extend our simulation to include
the $x$-position of the gas molecules.  To keep things simple, we will
simulate the motion along the $x$-axis after we have simulated
collisions and reached thermal equilibrium.

Start by generating \pyth{NGAS} $x$-positions uniformly from 0 to $L$.
Set the total momentum transferred ($\Delta p$) to zero.  During each time step,
update each $x$ value to $x + \tau \, v_x$.  For each molecule outside
the boundaries:
\begin{itemize}
 \item For $x<0$ set $x \to -x$
 \item For $x>L$ set $x \to 2L-x$
 \item Or handle both cases at once with:  $x \to (2L-x) \bmod L$
 \item Add $2 M |p_x|$ to the total momentum transferred $\Delta p$
 \item Set $v_x \to -v_x$ 
\end{itemize}   
After $N$ time steps of size $\tau$, the average force on each wall wil be given by:
\begin{displaymath}
F = \frac{\Delta p}{2 N \tau}
\end{displaymath}
If you use boolean masks effectively, you can implement this algorithm
with only a single for loop, which is needed to apply the $N$ time
steps.\\

\plot (Optional Challenge) Calculate the outward force on the wall of
the container for $L = 1 L_0$ and $T = 1 T_0$.  Compare to what you
expect from the ideal gas law.\\

\plot (Optional Challenge) Perform your Fource calculation {\em
  before} the gas reaches thermal equilirium.  Does the ideal gas law
require that the gas has reached thermal equilibrium?\\

\plot (Optional Challenge) Plot the simulated outward force on the
wall of container for several different $L$ values.  On the same axis,
plot the expected force as a function of $L$ as well.\\

\plot  (Optional Challenge) Plot the simulated outward force on the
wall of container for several different $T$ values.  On the same axis,
plot the expected force as a function of $T$ as well.\\


\chapter{Animation of Planetery Motion}

\section{Introduction}

In lecture, we applied the Verlet method to simulate planetery motion
in the presence of massive sun, and produced an animation showing the
trajectory of the planet.  There is no assignment in this chapter.
This is just to record the outcome of our exercise, for fun.

To use animation in our notebook, we cannot use the inline option for
plotting, so we start things off in the notebook style instead:
\begin{python}
  %pylab notebook
\end{python}

We worked through a simple example of an animated function:
\begin{python}
from matplotlib.animation import FuncAnimation

x = np.linspace(0, 2 * np.pi, 100)
ya = np.cos(x)
yb = np.sin(x)

plt.xlim(0,2*np.pi)
plt.ylim(-1.5,1.5)

la, = plt.plot([], [], "b-")
lb, = plt.plot([], [], "r--")

def animate(i):
    la.set_data(x[:i],ya[:i])
    lb.set_data(x[:i],yb[:i])
    
anim = FuncAnimation(plt.gcf(), animate, frames=100, interval=10, repeat=True)
plt.show()
\end{python}
Notice the use of the \pyth{set_data} function to update the plotted figures
la and lb.  You have a lot of options for how to handle this.

Next we implemented the Verlet method for a two-d system as:

\begin{python}
def verlet(tau,xa,xb,ya,yb,ax,ay):
    txa = xa
    tya = ya    
    xa = 2*xa-xb+tau**2*ax
    ya = 2*ya-yb+tau**2*ay    
    return xa,txa,ya,tya
\end{python}

And simulated the planetary motion as:
\begin{python}
vz = 1.0 
tau = 0.01
xa = 1
xb = 1
ya = 0
yb = -vz*tau
az = 1
tx = np.array([])
ty = np.array([])

for i in range(int(100/tau)):
    tx = np.append(tx,xa)
    ty = np.append(ty,ya)    
    ax = -az*xa/np.sqrt(xa**2+ya**2)**3
    ay = -az*ya/np.sqrt(xa**2+ya**2)**3    
    xa,xb,ya,yb = verlet(tau,xa,xb,ya,yb,ax,ay)
plt.xlim(-2,2)
plt.ylim(-2,2)
plt.plot(tx,ty)
\end{python}

With the trajectories saved as the arrays tx and ty, animation was produced like this:
\begin{python}
import matplotlib.pyplot as plt
from matplotlib.animation import FuncAnimation
import numpy as np
n = np.size(tx)
print(n)
plt.xlim(-2,2)
plt.ylim(-2,2)
plt.plot(0,0,"r+")
la, = plt.plot([], [], "bo")

def animate(i):
    la.set_data(tx[i],ty[i])
    
anim = FuncAnimation(plt.gcf(), animate, frames=n, interval=1, repeat=True)
plt.show()
\end{python}

















%\appendix

%\chapter{Debugging}

In our context, debugging is the process of finding and removing
mistakes, called bugs, from your software.  Singling this process out
is a bit deceptive, it makes it seems distinct from software
development, as if you should write your software, and then debug it.
Indeed many students start this way, but it is a painful and
ineffective approach.  Experienced programs debug {\em while} developing
their code.

The fundamental approach to debugging (which works equally well
outside of programming) is to break every problem down into simple,
well defined parts, and then thoroughly test each part.  When one part
does not work, you break it down into smaller parts.  This process can
be quite simple, such as adding print statements to each step of a
complicated calculation.  It can also be quite advanced, such as when
teams of experienced software developers use automated builds and a
suite of integration tests that validate every proposed change to code
before it is accepted.  Experienced programs still produce bugs, they
just get better at squashing them.

There are a number of well-loved techniques to debugging:
\begin{itemize}
\item Print statements.
\item Start with a simple problem.
\item Test on special cases.
\item Use paper and pencil.
\item Decrease the size.
\item Establish feedback.
\item Write modular code.
\item Maintain unit tests.
\end{itemize}


\end{document}


